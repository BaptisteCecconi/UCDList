\documentclass[11pt,a4paper]{ivoa}
\input tthdefs
   
\usepackage{hyperref}
\usepackage{verbatim}
\usepackage{longtable}
\usepackage{natbib} 
\usepackage{enumitem} 
\title{UCD1+ controlled vocabulary\\Updated List of Terms}

% see ivoatexDoc for what group names to use here
\ivoagroup{Semantics}

\author{Andrea Preite Martinez}
\author{Mireille Louys}
\author{Baptiste Cecconi}
\author{S\'ebastien Derri\`ere}
\author{Fran\c cois Ochsenbein}
\author{St\'ephane Erard}

\editor[mailto:baptiste.cecconi@obspm.fr]{Baptiste Cecconi}

\previousversion[http://www.ivoa.net/documents/UCD1+/20180527/REC-UCDlist-1.3-20180527.pdf]{The UCD1+ 
controlled vocabulary 1.3}

\setcounter{secnumdepth}{5}  

\begin{document}
\begin{abstract}
This document describes the \emph{list of controlled terms} used to build the Unified Content Descriptors, Version 1+ (UCD1+). 
The document describing the UCD1+ can be found at the URL: \url{http://www.ivoa.net/Documents/latest/UCD.html}. This 
document reviews the structure of the UCD1+ and presents the current vocabulary.  

This version contains new UCD words discussed and accepted by the Semantics Working Group during the UCD list v1.3 RFM.

\end{abstract} 

\section{Definition of atoms and words}
A UCD is a string which contains textual tokens called `words', separated by semicolons(;). A word is composed of 
`atoms', separated by periods(.). The hierarchy is as follows: 
$$
\textrm{atoms} \rightarrow \textrm{words} \rightarrow \textrm{composed words}
$$
UCD1+ are either single words, or a composition of several words.

UCDs are ``controlled'' through a process defined in the IVOA. See \citet{std:UCD} and section \ref{sec:words} below. 
Control is exercised at the level of words (UCD1+) and at the level of the vocabulary (atoms) used to form words. A 
consistent list of atoms will be maintained, making sure that the same atom always means the same thing, even if used 
in combination with different other atoms. 

\subsection{Definition of atoms}
Atoms are defined following these guidelines:
\begin{enumerate}
\item Abbreviations are used in contexts where their meaning is unambiguous. ({\bf ra}, {\bf dec} are 
acceptable, but {\bf t} is ambiguous: {\bf time} and {\bf temperature} are used instead.)
\item Atoms are not hyphenated. The separation is marked by a capital letter to help readability 
(position angle = {\bf posAng}) unless the composed word has a well- known acronym (signal to noise ratio = 
{\bf snr}) or short form (standard deviation = {\bf stdev}). There are only two exceptions to this rule: 
(i) the X-ray band ({\bf em.X-ray}) and (ii) the frequency / wavelength intervals defining regions of the 
e.m. spectrum (e.g., {\bf em.radio.3-6GHz}).
\end{enumerate}

\subsection{Definition of words}
\label{sec:words}
The list of UCD1+ words presented in this document was initially generated applying the rules and 
recommendations of PR-UCD-20040823 to catalogues/tables in VizieR. The original motivation was to 
transform old UCD1 into an improved version, trying to build a list of combinations of new words that 
could describe all the existing UCD1 terms.

The list of UCD1+ words is maintained by the UCD Scientific Board, following the procedure defined 
in the UCD Recommendation document \citep{std:UCD} and described in detail in 
\citet{std:UCDMAINT-PR}\footnote{An earlier draft on UCD building, still at 
\url{http://www.ivoa.net/documents/PR/UCD/UCD-20040823.html} includes more details 
about the process of the change from the earlier ``UCD1'' standard, and may be of historical interest, or 
provide more rationale.}.

\section{The structure of the UCD1+ tree}
All existing UCD1+ words are grouped into 12 main categories. These categories are expressed by the 
first atom of the word, whose possible values are:
\begin{enumerate}
\item {\bf arith} (arithmetics)\\ This section includes concepts involving or indicating some 
mathematical operation performed on the primary `concept' or just the presence of an arithmetic 
factor or operator.
\item {\bf em} (electromagnetic spectrum)\\ This section describes the electromagnetic spectrum, 
either in a monochromatic way or in predefined intervals. The complete list of proposed bands (in 
seven classical regions of the electromagnetic spectrum: radio, millimetre, infrared, optical, 
ultraviolet, x-ray and gamma- ray), can be found in the document 
\url{https://wiki.ivoa.net/internal/IVOA/IvoaUCD/NoteEMSpectrum-20040520.html}
\item {\bf instr} (instrument)\\ This section includes all quantities related to astronomical 
instrumentation, e.g. detectors (plates, CCDs, etc.), spectrographs, and telescopes (including 
observatories or missions), etc.
\item {\bf meta} (metadata)\\ This section includes all the information that is not coming directly 
from a measurement, and information that could not be included in other sections.
\item {\bf obs} (observation)\\ In principle under this section should go all words describing an 
observation (the name of the observer or PI, the observing conditions, the name of the field). In 
practice, this section helps to identify concepts related to an observation process.
\item {\bf phot} (photometry)\\ All the words describing photometric measures are included in this 
section. The definitions distinguish between a flux density (flux per unit frequency interval), a 
flux density integrated over a given electromagnetic spectrum interval (flux if expressed linearly, 
mag if expressed by a log), or a flux expressed in counts/s (if the setup of the detector is photon 
counting observing mode). `Colors', which are differences of magnitudes (i.e. ratios of fluxes) 
measured in different bandpasses, are also included.
\item {\bf phys} (physics)\\ This section includes atomic and molecular data (mainly used for 
spectroscopy) and basic physical quantities (temperature, mass, gravity, luminosity, etc.)
\item {\bf pos} (positional data)\\ This section describes all quantities related to the position of 
an object on the sky:
\begin{itemize}
\item Angular coordinates, and projections from spherical to rectangular systems.
\item Angular measurements in general (the angular size of an object is in this section, its linear 
size is in the {\bf phys} section).
\item The World Coordinate System FITS keywords.
\end{itemize}
\item {\bf spect} (spectral data)\\ For historical reasons, photometric data taken in narrow spectral 
bands with instruments called spectrographs are classified as spectroscopic data. These definitions 
should not be confused with those in the {\bf em} category. {\bf em} represents the independent 
variable, or dispersion axis, and {\bf phot} and {\bf spect} describe the dependent measures like a 
flux under the {\bf phot} branch, and spectral measures spectral line physical features one can 
measure on a spectrum, for instance, under the {\bf spect} branch. 
\item {\bf src} (source)\\ This is a rather generic section, mainly devoted to source classifications.
Variability, orbital, and velocity data are also included in this section.
\item {\bf stat} (statistics)\\ This section includes statistical information on measurements.
\item {\bf time} (time)\\ Quantities related to time (age, date, period, etc.) are described in this 
section.
\end{enumerate}

\section{Combining UCD words}
\subsection{Goal}
Since their definition UCDs have been used in major catalogue archives, in the definition of various
VO protocols (SSA, SIAv2, SLAP, TAP ObsTAP, EPN-TAP, etc.) and used with success to provide semantic
annotation for a huge collection of table columns distributed in the astronomical community.

The list of terms has increased and the usage of UCD combination has become very common. This leads
to a richer set of rules in the assigning and checking tools developed at CDS with VO partners.

In order to keep the consistency in the UCD thesaurus, each rule is adjusted and weighted considering 
the physical usage of the quantities represented in table columns, so the pertinence increases with 
the context. Initially used for source catalogues (Vizier, Heasarc archives, etc.) in the first place, 
they are now also used in VOTable documents for planetary data \citep{wd:epntap,erard-vespa} and all sorts of metadata.

\subsection{Remarks on combination rules}
The combination rules have been defined in the first IVOA documents defining UCD concept
\citep{std:UCD}. They are exposed with a syntax tag given as a property of each UCD word 
and included in the list of UCD words. See Appendix \ref{sec:list} with the tags definitions on top.

They correspond to real usage of the terms in science publications and are attached to the description 
of catalogues’ column by experimented data scientists. UCD combination also reflects the catalogues 
build-up strategy. Errors and statistics, for instance, are provided with measurement values; measures 
and model comparison are evaluated with error fits, precision, etc. All the scientific knowledge helps 
to define appropriate UCD words combination.

The assigning tool proposed at \url{http://cds.u-strasbg.fr/UCD/cgi-bin/descr2ucd} is based on the
pragmatic encoding of physical quantities found in science papers and data attached to publications.

\appendix
\section{Current questions about combinations of UCDs}
\subsection{How do UCDs differ from structured descriptions?}

UCDs do not provide a structured representation of table content but the meaning or relative class 
concept known at the time for the astronomical speciality. Therefore, the structure of words and 
their rules for combination do not follow any object oriented paradigm, in contradiction to any 
reference to a data model item (Utype, VO-DML type /role definition), which are dependent of a 
defined and endorsed IVOA data model specification.

\subsection{P or S syntax code: Which is the most pertinent position for a UCD word?}
P, S and Q are the labels expressing in which position of a UCD expression a term can be used, P 
in first place, S as suffix, and Q in both allowed position: head and tail. The UCD list defines 
the recommended position for each word with some flexibility.

P is always what matters the most to describe a quantity, i.e., the kind of property that should be 
searched for in primary order, and the most relevant UCD words to represent a quantity.

S is the code for the qualifying part of the UCD, the secondary information appended to specify 
the first UCD term.

%All the codes are explained and given in the list of UCD terms in Appendix A. 
Examples:
\begin{itemize}
\item Give me all columns / all catalogues with a column having a magnitude in R: {\tt magnitude} 
is the primary concept and band R is the secondary concept, so the ucd to search for is 
{\tt phot.mag;em.opt.R}.
\item Give me all columns with an error on magnitude B: here we shall use a query with ucd 
equals to {\tt stat.error;phot.mag;em.opt.B}. Here the main concept attached to the column 
value is error, qualified by {\tt phot.mag}, itself qualified by {\tt em.opt.B}.
\end{itemize}
Concatenation can apply more than one time, depending on ordering rules. See \citet{std:UCD}, section 3.3, 
for other details.

\section{List of valid words}
\label{sec:list}

All words are preceded by a `syntax' code that can help in the process of building composed UCD1+.
\begin{enumerate}
\item ``P'' means that the word can only be used as ``primary'' or first word;
\item ``S'' stands for only secondary: the word cannot be used as the first word to describe a 
single quantity;
\item ``Q'' means that the word can be used indifferently as first or secondary word;

The following cases behave as Q prefix and can be combined  as primary or secondary.
They specialize the combination rules:  
	\begin{enumerate}[label*=\arabic*.]
	\item ``E'' means a photometric quantity, and can be followed by a word describing a part of 
	the electromagnetic spectrum;
	\item ``C'' is a colour index, and can be followed by two successive word describing a part of 
	the electromagnetic spectrum;
	\item ``V'' stands for vector. Such a word can be followed by another describing the axis or 
	reference frame in which the measurement is done.
	\end{enumerate}
\end{enumerate}
\newpage

%\begin{table}[h!]
%\begin{tabular}{c|l|p{0.5\textwidth}}
\footnotesize\begin{longtable}[h!]{c|p{40ex}|p{0.45\textwidth}}
\sptablerule
\multicolumn{2}{l|}{\textbf{UCD word}}&\textbf{Description}\\
\sptablerule
% GENERATED: python3 convert_ucd_list_to_tex.py ucd-list-1_4.txt
Q & {\tt arith} & Arithmetic quantities\\
S & {\tt arith.diff} & Difference between two quantities described by the same UCD\\
P & {\tt arith.factor} & Numerical factor\\
P & {\tt arith.grad} & Gradient\\
P & {\tt arith.rate} & Rate (per time unit)\\
S & {\tt arith.ratio} & Ratio between two quantities described by the same UCD\\
S & {\tt arith.squared} & Squared quantity\\
S & {\tt arith.sum} & Summed or integrated quantity\\
S & {\tt arith.variation} & Generic variation of a quantity\\
Q & {\tt arith.zp} & Zero point\\
S & {\tt em} & Electromagnetic spectrum\\
S & {\tt em.IR} & Infrared part of the spectrum\\
S & {\tt em.IR.J} & Infrared between 1.0 and 1.5 micron\\
S & {\tt em.IR.H} & Infrared between 1.5 and 2 micron\\
S & {\tt em.IR.K} & Infrared between 2 and 3 micron\\
S & {\tt em.IR.3-4um} & Infrared between 3 and 4 micron\\
S & {\tt em.IR.4-8um} & Infrared between 4 and 8 micron\\
S & {\tt em.IR.8-15um} & Infrared between 8 and 15 micron\\
S & {\tt em.IR.15-30um} & Infrared between 15 and 30 micron\\
S & {\tt em.IR.30-60um} & Infrared between 30 and 60 micron\\
S & {\tt em.IR.60-100um} & Infrared between 60 and 100 micron\\
S & {\tt em.IR.NIR} & Near-Infrared, 1-5 microns\\
S & {\tt em.IR.MIR} & Medium-Infrared, 5-30 microns\\
S & {\tt em.IR.FIR} & Far-Infrared, 30-100 microns\\
S & {\tt em.UV} & Ultraviolet part of the spectrum\\
S & {\tt em.UV.10-50nm} & Ultraviolet between 10 and 50 nm EUV extreme UV\\
S & {\tt em.UV.50-100nm} & Ultraviolet between 50 and 100 nm\\
S & {\tt em.UV.100-200nm} & Ultraviolet between 100 and 200 nm FUV Far UV\\
S & {\tt em.UV.200-300nm} & Ultraviolet between 200 and 300 nm NUV near UV\\
S & {\tt em.X-ray} & X-ray part of the spectrum\\
S & {\tt em.X-ray.soft} & Soft X-ray (0.12 - 2 keV)\\
S & {\tt em.X-ray.medium} & Medium X-ray (2 - 12 keV)\\
S & {\tt em.X-ray.hard} & Hard X-ray (12 - 120 keV)\\
Q & {\tt em.bin} & Channel / instrumental spectral bin coordinate (bin number)\\
Q & {\tt em.energy} & Energy value in the em frame\\
Q & {\tt em.freq} & Frequency value in the em frame\\
Q & {\tt em.freq.cutoff} & cutoff frequency\\
Q & {\tt em.freq.resonance} & resonance frequency\\
S & {\tt em.gamma} & Gamma rays part of the spectrum\\
S & {\tt em.gamma.soft} & Soft gamma ray (120 - 500 keV)\\
S & {\tt em.gamma.hard} & Hard gamma ray (>500 keV)\\
S & {\tt em.line} & Designation of major atomic lines\\
S & {\tt em.line.HI} & 21cm hydrogen line\\
S & {\tt em.line.Lyalpha} & H-Lyalpha line\\
S & {\tt em.line.Halpha} & H-alpha line\\
S & {\tt em.line.Hbeta} & H-beta line\\
S & {\tt em.line.Hgamma} & H-gamma line\\
S & {\tt em.line.Hdelta} & H-delta line\\
S & {\tt em.line.Brgamma} & Bracket gamma line\\
S & {\tt em.line.OIII} & [OIII] line whose rest wl is 500.7 nm\\
S & {\tt em.line.CO} & CO radio line,  e.g 12CO(1-0) at 115GHz\\
S & {\tt em.mm} & Millimetric/submillimetric part of the spectrum\\
S & {\tt em.mm.30-50GHz} & Millimetric between 30 and 50 GHz\\
S & {\tt em.mm.50-100GHz} & Millimetric between 50 and 100 GHz\\
S & {\tt em.mm.100-200GHz} & Millimetric between 100 and 200 GHz\\
S & {\tt em.mm.200-400GHz} & Millimetric between 200 and 400 GHz\\
S & {\tt em.mm.400-750GHz} & Millimetric between 400 and 750 GHz\\
S & {\tt em.mm.750-1500GHz} & Millimetric between 750 and 1500 GHz\\
S & {\tt em.mm.1500-3000GHz} & Millimetric between 1500 and 3000 GHz\\
S & {\tt em.opt} & Optical part of the spectrum\\
S & {\tt em.opt.U} & Optical band between 300 and 400 nm\\
S & {\tt em.opt.B} & Optical band between 400 and 500 nm\\
S & {\tt em.opt.V} & Optical band between 500 and 600 nm\\
S & {\tt em.opt.R} & Optical band between 600 and 750 nm\\
S & {\tt em.opt.I} & Optical band between 750 and 1000 nm\\
S & {\tt em.pw} & Plasma waves (trapped in local medium)\\
S & {\tt em.radio} & Radio part of the spectrum\\
S & {\tt em.radio.20MHz} & Radio below 20 MHz\\
S & {\tt em.radio.20-100MHz} & Radio between 20 and 100 MHz\\
S & {\tt em.radio.100-200MHz} & Radio between 100 and 200 MHz\\
S & {\tt em.radio.200-400MHz} & Radio between 200 and 400 MHz\\
S & {\tt em.radio.400-750MHz} & Radio between 400 and 750 MHz\\
S & {\tt em.radio.750-1500MHz} & Radio between 750 and 1500 MHz\\
S & {\tt em.radio.1500-3000MHz} & Radio between 1500 and 3000 MHz\\
S & {\tt em.radio.3-6GHz} & Radio between 3 and 6 GHz\\
S & {\tt em.radio.6-12GHz} & Radio between 6 and 12 GHz\\
S & {\tt em.radio.12-30GHz} & Radio between 12 and 30 GHz\\
Q & {\tt em.wavenumber} & Wavenumber value in the em frame\\
Q & {\tt em.wl} & Wavelength value in the em frame\\
Q & {\tt em.wl.central} & Central wavelength\\
Q & {\tt em.wl.effective} & Effective wavelength\\
Q & {\tt instr} & Instrument\\
E & {\tt instr.background} & Instrumental background\\
Q & {\tt instr.bandpass} & Bandpass (e.g.: band name) of instrument\\
Q & {\tt instr.bandwidth} & Bandwidth of the instrument\\
Q & {\tt instr.baseline} & Baseline for interferometry\\
S & {\tt instr.beam} & Beam\\
Q & {\tt instr.calib} & Calibration parameter\\
S & {\tt instr.det} & Detector\\
Q & {\tt instr.det.noise} & Instrument noise\\
Q & {\tt instr.det.psf} & Point Spread Function\\
Q & {\tt instr.det.qe} & Quantum efficiency\\
Q & {\tt instr.dispersion} & Dispersion of a spectrograph\\
Q & {\tt instr.experiment} & Experiment or group of instruments\\
S & {\tt instr.filter} & Filter\\
S & {\tt instr.fov} & Field of view\\
S & {\tt instr.obsty} & Observatory, satellite, mission\\
Q & {\tt instr.obsty.seeing} & Seeing\\
Q & {\tt instr.offset} & Offset angle respect to main direction of observation\\
Q & {\tt instr.order} & Spectral order in a spectrograph\\
Q & {\tt instr.param} & Various instrumental parameters\\
S & {\tt instr.pixel} & Pixel (default size: angular)\\
S & {\tt instr.plate} & Photographic plate\\
Q & {\tt instr.plate.emulsion} & Plate emulsion\\
Q & {\tt instr.precision} & Instrument precision\\
Q & {\tt instr.rmsf} & Rotation Measure Spread Function\\
Q & {\tt instr.saturation} & Instrument saturation threshold\\
Q & {\tt instr.scale} & Instrument scale (for CCD, plate, image)\\
Q & {\tt instr.sensitivity} & Instrument sensitivity, detection threshold\\
Q & {\tt instr.setup} & Instrument configuration or setup\\
Q & {\tt instr.skyLevel} & Sky level\\
Q & {\tt instr.skyTemp} & Sky temperature\\
Q & {\tt instr.tel} & Telescope\\
Q & {\tt instr.tel.focalLength} & Telescope focal length\\
S & {\tt instr.voxel} & Related to a voxel (n-D volume element with n>2)\\
P & {\tt meta} & Metadata\\
P & {\tt meta.abstract} & Abstract (of paper, proposal, etc.)\\
P & {\tt meta.bib} & Bibliographic reference\\
P & {\tt meta.bib.author} & Author name\\
P & {\tt meta.bib.bibcode} & Bibcode\\
P & {\tt meta.bib.fig} & Figure in a paper\\
P & {\tt meta.bib.journal} & Journal name\\
P & {\tt meta.bib.page} & Page number\\
P & {\tt meta.bib.volume} & Volume number\\
Q & {\tt meta.calibLevel} & Processing/calibration level\\
Q & {\tt meta.checksum} & Numerical signature of digital data\\
P & {\tt meta.code} & Code or flag\\
P & {\tt meta.code.class} & Classification code\\
P & {\tt meta.code.error} & Limit uncertainty error flag\\
P & {\tt meta.code.member} & Membership code\\
P & {\tt meta.code.mime} & MIME type\\
P & {\tt meta.code.multip} & Multiplicity or binarity flag\\
P & {\tt meta.code.qual} & Quality, precision, reliability flag or code\\
P & {\tt meta.code.status} & Status code (e.g.: status of a proposal/observation)\\
P & {\tt meta.cryptic} & Unknown or impossible to understand quantity\\
P & {\tt meta.curation} & Identity of man/organization responsible for the data\\
Q & {\tt meta.dataset} & Dataset\\
Q & {\tt meta.email} & Curation/contact e-mail\\
S & {\tt meta.file} & File\\
S & {\tt meta.fits} & FITS standard\\
P & {\tt meta.id} & Identifier, name or designation\\
P & {\tt meta.id.assoc} & Identifier of associated counterpart\\
P & {\tt meta.id.CoI} & Name of Co-Investigator\\
P & {\tt meta.id.cross} & Cross identification\\
P & {\tt meta.id.parent} & Identification of parent source\\
P & {\tt meta.id.part} & Part of identifier, suffix or sub-component\\
P & {\tt meta.id.PI} & Name of Principal Investigator or Co-PI\\
S & {\tt meta.main} & Main value of something\\
S & {\tt meta.modelled} & Quantity was produced by a model\\
P & {\tt meta.note} & Note or remark (longer than a code or flag)\\
P & {\tt meta.number} & Number (of things; e.g. nb of object in an image)\\
S & {\tt meta.preview} & Related to a preview operation for a dataset\\
Q & {\tt meta.query} & A query posed to an information system or database or a property of it\\
P & {\tt meta.record} & Record number\\
P & {\tt meta.ref} & Reference or origin\\
P & {\tt meta.ref.doi} & DOI identifier (dereferenceable)\\
Q & {\tt meta.ref.ivoid} & Identifier as recommended  in the IVOA  (dereferenceable)\\
P & {\tt meta.ref.ivorn} & Identifier defined as IVORN, VO Resource Name (ivo://)  (deprecated)\\
P & {\tt meta.ref.uri} & URI, universal resource identifier\\
P & {\tt meta.ref.url} & URL, web address\\
S & {\tt meta.software} & Software used in generating data\\
S & {\tt meta.table} & Table or catalogue\\
P & {\tt meta.title} & Title or explanation\\
Q & {\tt meta.ucd} & UCD\\
P & {\tt meta.unit} & Unit\\
P & {\tt meta.version} & Version\\
S & {\tt obs} & Observation\\
Q & {\tt obs.airMass} & Airmass\\
S & {\tt obs.atmos} & Atmosphere, atmospheric phenomena affecting an observation\\
Q & {\tt obs.atmos.extinction} & Atmospheric extinction\\
Q & {\tt obs.atmos.refractAngle} & Atmospheric refraction angle\\
S & {\tt obs.calib} & Calibration observation\\
S & {\tt obs.calib.flat} & Related to flat-field calibration observation (dome, sky, ..)\\
S & {\tt obs.calib.dark} & Related to dark current calibration\\
S & {\tt obs.exposure} & Exposure\\
S & {\tt obs.field} & Region covered by the observation\\
S & {\tt obs.image} & Image\\
Q & {\tt obs.observer} & Observer, discoverer\\
S & {\tt obs.occult} & Observation of occultation phenomenon by solar system objects\\
S & {\tt obs.transit} & Observation of transit phenomenon  : exo-planets\\
Q & {\tt obs.param} & Various observation or reduction parameter\\
S & {\tt obs.proposal} & Observation proposal\\
Q & {\tt obs.proposal.cycle} & Proposal cycle\\
S & {\tt obs.sequence} & Sequence of observations, exposures or events\\
E & {\tt phot} & Photometry\\
E & {\tt phot.antennaTemp} & Antenna temperature\\
Q & {\tt phot.calib} & Photometric calibration\\
C & {\tt phot.color} & Color index or magnitude difference\\
Q & {\tt phot.color.excess} & Color excess\\
Q & {\tt phot.color.reddFree} & Dereddened color\\
E & {\tt phot.count} & Flux expressed in counts\\
E & {\tt phot.fluence} & Radiant photon energy received by a surface per unit area or irradiance of a surface integrated over time of irradiation\\
E & {\tt phot.flux} & Photon flux or irradiance\\
Q & {\tt phot.flux.bol} & Bolometric flux\\
E & {\tt phot.flux.density} & Flux density (per wl/freq/energy interval)\\
E & {\tt phot.flux.density.sb} & Flux density surface brightness\\
E & {\tt phot.flux.sb} & Flux surface brightness\\
E & {\tt phot.limbDark} & Limb-darkening coefficients\\
E & {\tt phot.mag} & Photometric magnitude\\
E & {\tt phot.mag.bc} & Bolometric correction\\
Q & {\tt phot.mag.bol} & Bolometric magnitude\\
Q & {\tt phot.mag.distMod} & Distance modulus\\
E & {\tt phot.mag.reddFree} & Dereddened magnitude\\
E & {\tt phot.mag.sb} & Surface brightness in magnitude units\\
E & {\tt phot.radiance} & Radiance as energy flux per solid angle\\
Q & {\tt phys} & Physical quantities\\
Q & {\tt phys.SFR} & Star formation rate\\
E & {\tt phys.absorption} & Extinction or absorption along the line of sight\\
Q & {\tt phys.absorption.coeff} & Absorption coefficient (e.g. in a spectral line)\\
Q & {\tt phys.absorption.gal} & Galactic extinction\\
Q & {\tt phys.absorption.opticalDepth} & Optical depth\\
Q & {\tt phys.abund} & Abundance\\
Q & {\tt phys.abund.Fe} & Fe/H abundance\\
Q & {\tt phys.abund.X} & Hydrogen abundance\\
Q & {\tt phys.abund.Y} & Helium abundance\\
Q & {\tt phys.abund.Z} & Metallicity abundance\\
Q & {\tt phys.acceleration} & Acceleration\\
S & {\tt phys.aerosol} & Relative to aerosol\\
Q & {\tt phys.albedo} & Albedo or reflectance\\
Q & {\tt phys.angArea} & Angular area\\
Q & {\tt phys.angMomentum} & Angular momentum\\
E & {\tt phys.angSize} & Angular size width diameter dimension extension major minor axis extraction radius\\
E & {\tt phys.angSize.smajAxis} & Angular size extent or extension of semi-major axis\\
E & {\tt phys.angSize.sminAxis} & Angular size extent or extension of semi-minor axis\\
Q & {\tt phys.area} & Area (in surface, not angular units)\\
S & {\tt phys.atmol} & Atomic and molecular physics (shared properties)\\
Q & {\tt phys.atmol.branchingRatio} & Branching ratio\\
S & {\tt phys.atmol.collisional} & Related to collisions\\
Q & {\tt phys.atmol.collStrength} & Collisional strength\\
Q & {\tt phys.atmol.configuration} & Configuration\\
Q & {\tt phys.atmol.crossSection} & Atomic / molecular cross-section\\
Q & {\tt phys.atmol.element} & Element\\
Q & {\tt phys.atmol.excitation} & Atomic molecular excitation parameter\\
Q & {\tt phys.atmol.final} & Quantity refers to atomic/molecular final/ground state, level, etc.\\
Q & {\tt phys.atmol.initial} & Quantity refers to atomic/molecular initial state, level, etc.\\
Q & {\tt phys.atmol.ionStage} & Ion, ionization stage\\
S & {\tt phys.atmol.ionization} & Related to ionization\\
Q & {\tt phys.atmol.lande} & Lande factor\\
S & {\tt phys.atmol.level} & Atomic level\\
Q & {\tt phys.atmol.lifetime} & Lifetime of a level\\
Q & {\tt phys.atmol.lineShift} & Line shifting coefficient\\
Q & {\tt phys.atmol.number} & Atomic number Z\\
Q & {\tt phys.atmol.oscStrength} & Oscillator strength\\
Q & {\tt phys.atmol.parity} & Parity\\
Q & {\tt phys.atmol.qn} & Quantum number\\
Q & {\tt phys.atmol.radiationType} & Type of radiation characterizing atomic lines (electric dipole/quadrupole, magnetic dipole)\\
Q & {\tt phys.atmol.symmetry} & Type of nuclear spin symmetry\\
Q & {\tt phys.atmol.sWeight} & Statistical weight\\
Q & {\tt phys.atmol.sWeight.nuclear} & Statistical weight for nuclear spin states\\
Q & {\tt phys.atmol.term} & Atomic term\\
S & {\tt phys.atmol.transition} & Transition between states\\
Q & {\tt phys.atmol.transProb} & Transition probability, Einstein A coefficient\\
Q & {\tt phys.atmol.wOscStrength} & Weighted oscillator strength\\
Q & {\tt phys.atmol.weight} & Atomic weight\\
Q & {\tt phys.columnDensity} & Column density\\
S & {\tt phys.composition} & Quantities related to composition of objects\\
Q & {\tt phys.composition.massLightRatio} & Mass to light ratio\\
Q & {\tt phys.composition.yield} & Mass yield\\
S & {\tt phys.cosmology} & Related to cosmology\\
Q & {\tt phys.current} & Electric current\\
Q & {\tt phys.current.density} & Electric current density\\
Q & {\tt phys.damping} & Generic damping quantities\\
Q & {\tt phys.density} & Density (of mass, electron, ...)\\
Q & {\tt phys.density.phaseSpace} & Density in the phase space\\
Q & {\tt phys.dielectric} & Complex dielectric function\\
Q & {\tt phys.dispMeasure} & Dispersion measure\\
S & {\tt phys.dust} & Relative to dust\\
Q & {\tt phys.electCharge} & Electric charge\\
V & {\tt phys.electField} & Electric field\\
S & {\tt phys.electron} & Electron\\
Q & {\tt phys.electron.degen} & Electron degeneracy parameter\\
Q & {\tt phys.emissMeasure} & Emission measure\\
Q & {\tt phys.emissivity} & Emissivity\\
Q & {\tt phys.energy} & Energy\\
Q & {\tt phys.energy.Gibbs} & Gibbs (free) energy or free enthalpy   [ G=H –TS ]\\
Q & {\tt phys.energy.Helmholtz} & Helmholtz free energy [ A=U–TS ]\\
Q & {\tt phys.energy.density} & Energy density\\
Q & {\tt phys.enthalpy} & Enthalpy  [ H=U+pv ]\\
Q & {\tt phys.entropy} & Entropy\\
Q & {\tt phys.eos} & Equation of state\\
Q & {\tt phys.excitParam} & Excitation parameter U\\
E & {\tt phys.fluence} & Particle energy received  by a surface per unit area and integrated over time\\
Q & {\tt phys.flux} & Flux or flow of particle, energy, etc.\\
Q & {\tt phys.flux.energy} & Energy flux, heat flux\\
Q & {\tt phys.gauntFactor} & Gaunt factor/correction\\
Q & {\tt phys.gravity} & Gravity\\
Q & {\tt phys.ionizParam} & Ionization parameter\\
Q & {\tt phys.ionizParam.coll} & Collisional ionization\\
Q & {\tt phys.ionizParam.rad} & Radiative ionization\\
E & {\tt phys.luminosity} & Luminosity\\
Q & {\tt phys.luminosity.fun} & Luminosity function\\
E & {\tt phys.magAbs} & Absolute magnitude\\
Q & {\tt phys.magAbs.bol} & Bolometric absolute magnitude\\
V & {\tt phys.magField} & Magnetic field\\
Q & {\tt phys.mass} & Mass\\
Q & {\tt phys.mass.inertiaMomentum} & Momentum of inertia or rotational inertia\\
Q & {\tt phys.mass.loss} & Mass loss\\
Q & {\tt phys.mol} & Molecular data\\
Q & {\tt phys.mol.dipole} & Molecular dipole\\
Q & {\tt phys.mol.dipole.electric} & Molecular electric dipole moment\\
Q & {\tt phys.mol.dipole.magnetic} & Molecular magnetic dipole moment\\
Q & {\tt phys.mol.dissociation} & Molecular dissociation\\
Q & {\tt phys.mol.formationHeat} & Formation heat for molecules\\
Q & {\tt phys.mol.quadrupole} & Molecular quadrupole\\
Q & {\tt phys.mol.quadrupole.electric} & Molecular electric quadrupole moment\\
S & {\tt phys.mol.rotation} & Molecular rotation\\
S & {\tt phys.mol.vibration} & Molecular vibration\\
S & {\tt phys.particle} & Related to physical particles\\
S & {\tt phys.particle.neutrino} & Related to neutrino\\
S & {\tt phys.particle.neutron} & Related to neutron\\
S & {\tt phys.particle.proton} & Related to proton\\
S & {\tt phys.particle.alpha} & Related to alpha particle\\
S & {\tt phys.phaseSpace} & Related to phase space\\
E & {\tt phys.polarization} & Polarization degree (or percentage)\\
Q & {\tt phys.polarization.circular} & Circular polarization\\
Q & {\tt phys.polarization.coherency} & Matrix of the correlation between components of an electromagnetic wave\\
Q & {\tt phys.polarization.linear} & Linear polarization\\
Q & {\tt phys.polarization.rotMeasure} & Rotation measure polarization\\
Q & {\tt phys.polarization.stokes} & Stokes polarization\\
Q & {\tt phys.polarization.stokes.I} & Stokes polarization coefficient I\\
Q & {\tt phys.polarization.stokes.Q} & Stokes polarization coefficient Q\\
Q & {\tt phys.polarization.stokes.U} & Stokes polarization coefficient U\\
Q & {\tt phys.polarization.stokes.V} & Stokes polarization coefficient V\\
Q & {\tt phys.potential} & Potential (electric, gravitational, etc.)\\
Q & {\tt phys.pressure} & Pressure\\
Q & {\tt phys.recombination.coeff} & Recombination coefficient\\
Q & {\tt phys.reflectance} & Radiance factor (received radiance divided by input radiance)\\
Q & {\tt phys.reflectance.bidirectional} & Bidirectional reflectance\\
Q & {\tt phys.reflectance.bidirectional.df} & Bidirectional reflectance distribution function\\
Q & {\tt phys.reflectance.factor} & Reflectance normalized per direction cosine of incidence angle\\
Q & {\tt phys.refractIndex} & Refraction index\\
Q & {\tt phys.size} & Linear size, length (not angular)\\
Q & {\tt phys.size.axisRatio} & Axis ratio (a/b) or (b/a)\\
Q & {\tt phys.size.diameter} & Diameter\\
Q & {\tt phys.size.radius} & Radius\\
Q & {\tt phys.size.smajAxis} & Linear semi major axis\\
Q & {\tt phys.size.sminAxis} & Linear semi minor axis\\
Q & {\tt phys.size.smedAxis} & Linear semi median axis for 3D ellipsoids\\
Q & {\tt phys.temperature} & Temperature\\
Q & {\tt phys.temperature.effective} & Effective temperature\\
Q & {\tt phys.temperature.electron} & Electron temperature\\
Q & {\tt phys.transmission} & Transmission (of filter, instrument, ...)\\
V & {\tt phys.veloc} & Space velocity\\
Q & {\tt phys.veloc.ang} & Angular velocity\\
Q & {\tt phys.veloc.dispersion} & Velocity dispersion\\
Q & {\tt phys.veloc.escape} & Escape velocity\\
Q & {\tt phys.veloc.expansion} & Expansion velocity\\
Q & {\tt phys.veloc.microTurb} & Microturbulence velocity\\
Q & {\tt phys.veloc.orbital} & Orbital velocity\\
Q & {\tt phys.veloc.pulsat} & Pulsational velocity\\
Q & {\tt phys.veloc.rotat} & Rotational velocity\\
Q & {\tt phys.veloc.transverse} & Transverse / tangential velocity\\
S & {\tt phys.virial} & Related to virial quantities (mass, radius, ..)\\
Q & {\tt phys.volume} & Volume (in cubic units)\\
Q & {\tt pos} & Position and coordinates\\
Q & {\tt pos.angDistance} & Angular distance, elongation\\
Q & {\tt pos.angResolution} & Angular resolution\\
Q & {\tt pos.az} & Position in alt-azimutal frame\\
Q & {\tt pos.az.alt} & Alt-azimutal altitude\\
Q & {\tt pos.az.azi} & Alt-azimutal azimut\\
Q & {\tt pos.az.zd} & Alt-azimutal zenith distance\\
Q & {\tt pos.azimuth} & Azimuthal angle in a generic reference plane\\
S & {\tt pos.barycenter} & Barycenter\\
S & {\tt pos.bodycentric} & Body-centric related coordinate\\
S & {\tt pos.bodygraphic} & Body-graphic related coordinate\\
S & {\tt pos.bodyrc} & Body related coordinates\\
Q & {\tt pos.bodyrc.alt} & Body related coordinate (altitude on the body)\\
Q & {\tt pos.bodyrc.lat} & Body related coordinate (latitude on the body)\\
Q & {\tt pos.bodyrc.lon} & Body related coordinate (longitude on the body)\\
S & {\tt pos.cartesian} & Cartesian (rectangular) coordinates\\
Q & {\tt pos.cartesian.x} & Cartesian coordinate along the x-axis\\
Q & {\tt pos.cartesian.y} & Cartesian coordinate along the y-axis\\
Q & {\tt pos.cartesian.z} & Cartesian coordinate along the z-axis\\
S & {\tt pos.centroid} & Related to centroid position\\
S & {\tt pos.cmb} & Cosmic Microwave Background reference frame\\
S & {\tt pos.cylindrical} & Related to cylindrical coordinates\\
Q & {\tt pos.cylindrical.azi} & Azimuthal angle around z-axis (cylindrical coordinates)\\
Q & {\tt pos.cylindrical.r} & Radial distance from z-axis (cylindrical coordinates)\\
Q & {\tt pos.cylindrical.z} & Height or altitude from reference plane (cylindrical coordinates)\\
Q & {\tt pos.dirCos} & Direction cosine\\
V & {\tt pos.distance} & Linear distance\\
S & {\tt pos.earth} & Coordinates related to Earth\\
Q & {\tt pos.earth.altitude} & Altitude, height on Earth  above sea level\\
Q & {\tt pos.earth.lat} & Latitude on Earth\\
Q & {\tt pos.earth.lon} & Longitude on Earth\\
S & {\tt pos.ecliptic} & Ecliptic coordinates\\
Q & {\tt pos.ecliptic.lat} & Ecliptic latitude\\
Q & {\tt pos.ecliptic.lon} & Ecliptic longitude\\
Q & {\tt pos.emergenceAng} & Emergence angle of optical ray on an interface\\
S & {\tt pos.eop} & Earth orientation parameters\\
Q & {\tt pos.ephem} & Ephemeris\\
Q & {\tt pos.eq} & Equatorial coordinates\\
Q & {\tt pos.eq.dec} & Declination in equatorial coordinates\\
Q & {\tt pos.eq.ha} & Hour-angle\\
Q & {\tt pos.eq.ra} & Right ascension in equatorial coordinates\\
Q & {\tt pos.eq.spd} & South polar distance in equatorial coordinates\\
S & {\tt pos.errorEllipse} & Positional error ellipse\\
Q & {\tt pos.frame} & Reference frame used for positions\\
S & {\tt pos.galactic} & Galactic coordinates\\
Q & {\tt pos.galactic.lat} & Latitude in galactic coordinates\\
Q & {\tt pos.galactic.lon} & Longitude in galactic coordinates\\
S & {\tt pos.galactocentric} & Galactocentric coordinate system\\
S & {\tt pos.geocentric} & Geocentric coordinate system\\
Q & {\tt pos.healpix} & Hierarchical Equal Area IsoLatitude Pixelization\\
S & {\tt pos.heliocentric} & Heliocentric position coordinate (solar system bodies)\\
Q & {\tt pos.HTM} & Hierarchical Triangular Mesh\\
Q & {\tt pos.incidenceAng} & Incidence angle of optical ray on an interface\\
S & {\tt pos.lambert} & Lambert projection\\
S & {\tt pos.lg} & Local Group reference frame\\
S & {\tt pos.lsr} & Local Standard of Rest reference frame\\
Q & {\tt pos.lunar} & Lunar coordinates\\
Q & {\tt pos.lunar.occult} & Occultation by lunar limb\\
Q & {\tt pos.nutation} & Nutation (of a body)\\
Q & {\tt pos.outline} & Set of points outlining a region (contour)\\
Q & {\tt pos.parallax} & Parallax\\
Q & {\tt pos.parallax.dyn} & Dynamical parallax\\
Q & {\tt pos.parallax.phot} & Photometric parallaxes\\
Q & {\tt pos.parallax.spect} & Spectroscopic parallax\\
Q & {\tt pos.parallax.trig} & Trigonometric parallax\\
Q & {\tt pos.phaseAng} & Phase angle, e.g. elongation of earth from sun as seen from a third cel. object\\
V & {\tt pos.pm} & Proper motion\\
Q & {\tt pos.posAng} & Position angle of a given vector\\
V & {\tt pos.precess} & Precession (in equatorial coordinates)\\
Q & {\tt pos.resolution} & Spatial linear resolution (not angular)\\
S & {\tt pos.spherical} & Related to spherical coordinates\\
Q & {\tt pos.spherical.azi} & Azimuthal angle (spherical coordinates)\\
Q & {\tt pos.spherical.colat} & Polar or Colatitude angle (spherical coordinates)\\
Q & {\tt pos.spherical.r} & Radial distance or radius (spherical coordinates)\\
S & {\tt pos.supergalactic} & Supergalactic coordinates\\
Q & {\tt pos.supergalactic.lat} & Latitude in supergalactic coordinates\\
Q & {\tt pos.supergalactic.lon} & Longitude in supergalactic coordinates\\
P & {\tt pos.wcs} & WCS keywords\\
P & {\tt pos.wcs.cdmatrix} & WCS CDMATRIX\\
P & {\tt pos.wcs.crpix} & WCS CRPIX\\
P & {\tt pos.wcs.crval} & WCS CRVAL\\
P & {\tt pos.wcs.ctype} & WCS CTYPE\\
P & {\tt pos.wcs.naxes} & WCS NAXES\\
P & {\tt pos.wcs.naxis} & WCS NAXIS\\
P & {\tt pos.wcs.scale} & WCS scale or scale of an image\\
Q & {\tt spect} & Spectroscopy\\
Q & {\tt spect.binSize} & Spectral bin size\\
S & {\tt spect.continuum} & Continuum spectrum\\
Q & {\tt spect.dopplerParam} & Doppler parameter b\\
E & {\tt spect.dopplerVeloc} & Radial velocity, derived from the shift of some spectral feature\\
E & {\tt spect.dopplerVeloc.opt} & Radial velocity derived from a wavelength shift using the optical convention\\
E & {\tt spect.dopplerVeloc.radio} & Radial velocity derived from a frequency shift using the radio convention\\
E & {\tt spect.index} & Spectral index\\
S & {\tt spect.line} & Spectral line\\
E & {\tt spect.line.asymmetry} & Line asymmetry\\
E & {\tt spect.line.broad} & Spectral line broadening\\
Q & {\tt spect.line.broad.Stark} & Stark line broadening coefficient\\
E & {\tt spect.line.broad.Zeeman} & Zeeman broadening\\
E & {\tt spect.line.eqWidth} & Line equivalent width\\
E & {\tt spect.line.intensity} & Line intensity\\
E & {\tt spect.line.profile} & Line profile\\
Q & {\tt spect.line.strength} & Spectral line strength S\\
E & {\tt spect.line.width} & Spectral line full width half maximum\\
Q & {\tt spect.resolution} & Spectral (or velocity) resolution\\
S & {\tt src} & Observed source viewed on the sky\\
S & {\tt src.calib} & Calibration source\\
S & {\tt src.calib.guideStar} & Guide star\\
Q & {\tt src.class} & Source classification (star, galaxy, cluster, comet, asteroid )\\
Q & {\tt src.class.color} & Color classification\\
Q & {\tt src.class.distance} & Distance class e.g. Abell\\
Q & {\tt src.class.luminosity} & Luminosity class\\
Q & {\tt src.class.richness} & Richness class e.g. Abell\\
Q & {\tt src.class.starGalaxy} & Star/galaxy discriminator, stellarity index\\
Q & {\tt src.class.struct} & Structure classification e.g. Bautz-Morgan\\
Q & {\tt src.density} & Density of sources\\
Q & {\tt src.ellipticity} & Source ellipticity\\
Q & {\tt src.impactParam} & Impact parameter\\
Q & {\tt src.morph} & Morphology structure\\
Q & {\tt src.morph.param} & Morphological parameter\\
Q & {\tt src.morph.scLength} & Scale length for a galactic component (disc or bulge)\\
Q & {\tt src.morph.type} & Hubble morphological type (galaxies)\\
S & {\tt src.net} & Qualifier indicating that a quantity (e.g. flux) is background subtracted rather than total\\
Q & {\tt src.orbital} & Orbital parameters\\
Q & {\tt src.orbital.eccentricity} & Orbit eccentricity\\
Q & {\tt src.orbital.inclination} & Orbit inclination\\
Q & {\tt src.orbital.meanAnomaly} & Orbit mean anomaly\\
Q & {\tt src.orbital.meanMotion} & Mean motion\\
Q & {\tt src.orbital.node} & Ascending node\\
Q & {\tt src.orbital.periastron} & Periastron\\
Q & {\tt src.orbital.Tisserand} & Tisserand parameter (generic)\\
Q & {\tt src.orbital.TissJ} & Tisserand parameter with respect to Jupiter\\
Q & {\tt src.redshift} & Redshift\\
Q & {\tt src.redshift.phot} & Photometric redshift\\
Q & {\tt src.sample} & Sample\\
Q & {\tt src.spType} & Spectral type MK\\
Q & {\tt src.var} & Variability of source\\
E & {\tt src.var.amplitude} & Amplitude of variation\\
Q & {\tt src.var.index} & Variability index\\
Q & {\tt src.var.pulse} & Pulse\\
Q & {\tt stat} & Statistical parameters\\
Q & {\tt stat.asymmetry} & Measure of asymmetry\\
P & {\tt stat.correlation} & Correlation between two parameters\\
P & {\tt stat.covariance} & Covariance between two parameters\\
P & {\tt stat.error} & Statistical error\\
P & {\tt stat.error.sys} & Systematic error\\
Q & {\tt stat.filling} & Filling factor (volume, time, ..)\\
Q & {\tt stat.fit} & Fit\\
P & {\tt stat.fit.chi2} & Chi2\\
P & {\tt stat.fit.dof} & Degrees of freedom\\
P & {\tt stat.fit.goodness} & Goodness or significance of fit\\
S & {\tt stat.fit.omc} & Observed minus computed\\
Q & {\tt stat.fit.param} & Parameter of fit\\
P & {\tt stat.fit.residual} & Residual fit\\
Q & {\tt stat.Fourier} & Fourier coefficient\\
Q & {\tt stat.Fourier.amplitude} & Amplitude of Fourier coefficient\\
S & {\tt stat.fwhm} & Full width at half maximum\\
S & {\tt stat.interval} & Generic interval between two limits (defined as a pair of values)\\
P & {\tt stat.likelihood} & Likelihood\\
S & {\tt stat.max} & Maximum or upper limit\\
S & {\tt stat.mean} & Mean, average value\\
S & {\tt stat.median} & Median value\\
S & {\tt stat.min} & Minimum or lowest limit\\
Q & {\tt stat.param} & Parameter\\
Q & {\tt stat.probability} & Probability\\
P & {\tt stat.rank} & Rank or order in list of sorted values\\
P & {\tt stat.rms} & Root mean square as square root of sum of squared values or quadratic mean\\
P & {\tt stat.snr} & Signal to noise ratio\\
P & {\tt stat.stdev} & Standard deviation as the square root of the variance\\
S & {\tt stat.uncalib} & Qualifier of a generic uncalibrated quantity\\
Q & {\tt stat.value} & Miscellaneous value\\
P & {\tt stat.variance} & Variance\\
P & {\tt stat.weight} & Statistical weight\\
Q & {\tt time} & Time, generic quantity in units of time or date\\
Q & {\tt time.age} & Age\\
Q & {\tt time.creation} & Creation time/date (of dataset, file, catalogue,...)\\
Q & {\tt time.crossing} & Crossing time\\
Q & {\tt time.duration} & Interval of time describing the duration of a generic event or phenomenon\\
Q & {\tt time.end} & End time/date of a generic event\\
Q & {\tt time.epoch} & Instant of time related to a generic event (epoch, date, Julian date, time stamp/tag,...)\\
Q & {\tt time.equinox} & Equinox\\
Q & {\tt time.interval} & Time interval, time-bin, time elapsed between two events, not the duration of an event\\
Q & {\tt time.lifetime} & Lifetime\\
Q & {\tt time.period} & Period, interval of time between the recurrence of phases in a periodic phenomenon\\
Q & {\tt time.period.revolution} & Period of revolution of a body around a primary one (similar to year)\\
Q & {\tt time.period.rotation} & Period of rotation of a body around its axis (similar to day)\\
Q & {\tt time.phase} & Phase, position within a period\\
Q & {\tt time.processing} & A time/date associated with the processing of data\\
Q & {\tt time.publiYear} & Publication year\\
Q & {\tt time.relax} & Relaxation time\\
Q & {\tt time.release} & The time/date data is available to the public\\
Q & {\tt time.resolution} & Time resolution\\
Q & {\tt time.scale} & Timescale\\
Q & {\tt time.start} & Start time/date of generic event\\

% /GENERATED
\sptablerule
%\end{tabular}
%\caption{List of UCD words}
\label{table:ucd-list}
\end{longtable}

\section{Changes from previous versions}

\subsection{Changes from REC v1.3 following RFM}

The modifications decided during the UCD1+ list v1.3 
RFM\footnote{\url{https://wiki.ivoa.net/twiki/bin/view/IVOA/UCDList_1-3_RFM}} process are presented below.

%\subsubsection*{Deletion}
%\subsubsection*{Ammendment}
\subsubsection*{Additions}
\footnotesize\begin{longtable}[h!]{c|p{40ex}|p{0.45\textwidth}}
\sptablerule
Q & {\tt phys.electCharge} & Electric charge\\
Q & {\tt phys.current} & Electric current\\
Q & {\tt phys.current.density} & Electric current density\\
Q & {\tt pos.incidenceAng} & Incidence angle of optical ray on an interface\\
Q & {\tt pos.emergenceAng} & Emergence angle of optical ray on an interface\\
Q & {\tt pos.azimuth} & azimuthal angle in a generic reference plane\\
Q & {\tt phys.reflectance} & Radiance factor (received radiance divided by input radiance)\\
Q & {\tt phys.reflectance.bidirectional} & Bidirectional reflectance\\
Q & {\tt phys.reflectance.bidirectional.df} & Bidirectional reflectance distribution function\\
Q & {\tt phys.reflectance.factor} & Reflectance normalized per direction cosine of incidence angle\\
S & {\tt pos.cylindrical} & Related to cylindrical coordinates\\
Q & {\tt pos.cylindrical.r} & Radial distance from z-axis (cylindrical coordinates)\\
Q & {\tt pos.cylindrical.azi} & Azimuthal angle around z-axis (cylindrical coordinates)\\
Q & {\tt pos.cylindrical.z} & Height or altitude from reference plane (cylindrical coordinates)\\
S & {\tt pos.spherical} & Related to spherical coordinates\\
Q & {\tt pos.spherical.r} & Radial distance or radius (spherical coordinates)\\
Q & {\tt pos.spherical.colat} & Polar or Colatitude angle (spherical coordinates)\\
Q & {\tt pos.spherical.azi} & Azimuthal angle (spherical coordinates)\\
Q & {\tt pos.resolution} & Spatial linear resolution (not angular)\\
S & {\tt pos.bodycentric} & Body-centric related coordinate\\
S & {\tt pos.bodygraphic} & Body-graphic related coordinate\\
Q & {\tt meta.checksum} & Numerical signature of digital data\\
Q & {\tt phys.polarization.coherency} & Matrix of the correlation between components of an electromagnetic wave\\
\sptablerule
\end{longtable}
\subsubsection*{Clarification}
%text improvement 
Clarified position rules for syntax code E, C, V in Appendix B.  

\subsection{Changes from PR v1.3-2018 following TCG comments}
\subsubsection*{Update of definitions} 
\footnotesize\begin{longtable}[h!]{c|p{40ex}|p{0.45\textwidth}}
\sptablerule
Q & {\tt meta.query} &  A query posed to an information system or database or a property of it\\
\sptablerule
\end{longtable}

\subsubsection*{Changes of position indicator} 
Was
\footnotesize\begin{longtable}[h!]{c|p{40ex}|p{0.45\textwidth}}
\sptablerule
Q & {\tt phys.atmol.collisional} & Related to collisions\\
Q & {\tt phys.virial}            & Related to virial quantities (mass, radius,..)\\
\sptablerule
\end{longtable}
Changed to S to conform to the “Related to” definition and the usage of this UCD, mostly appearing as suffix.  
\footnotesize\begin{longtable}[h!]{c|p{40ex}|p{0.45\textwidth}}
\sptablerule
S & {\tt phys.atmol.collisional} & Related to collisions\\
S & {\tt phys.virial}            & Related to virial quantities (mass, radius,..)\\
\sptablerule
\end{longtable}

\subsection{Changes from WD v1.3-20160719}
Added section 3 Remarks on combination rules for UCD words.

\subsubsection*{New terms}
\footnotesize\begin{longtable}[h!]{c|p{40ex}|p{0.45\textwidth}}
\sptablerule
P & {\tt meta.ref.doi} & DOI identifier (dereferenceable)\\
\sptablerule
\end{longtable}

\subsection{Changes from WD v1.23-20160719}
\subsubsection*{Additions}
\footnotesize\begin{longtable}[h!]{c|p{40ex}|p{0.45\textwidth}}
\sptablerule
S & {\tt arith.squared} & Squared quantity\\
S & {\tt arith.sum} & Summed or integrated quantity\\
S & {\tt arith.variation} & Generic variation of a quantity\\
S & {\tt instr.voxel} & Related to a voxel ( n-D volume element with n>2)\\
Q & {\tt pos.outline} & Set of points outlining a region (contour)\\
Q & {\tt stat.asymmetry} & Measure of asymmetry\\
Q & {\tt phys.polarization.stokes.I} & Stokes polarization coefficient I\\
Q & {\tt phys.polarization.stokes.Q} & Stokes polarization coefficient Q\\
Q & {\tt phys.polarization.stokes.U} & Stokes polarization coefficient U\\
Q & {\tt phys.polarization.stokes.V} & Stokes polarization coefficient V\\
Q & {\tt stat.asymmetry} & Measure of asymmetry\\
S & {\tt stat.fwhm} & Full width at half maximum\\
S & {\tt stat.interval} & Generic interval between two limits (defined as a pair of values)\\
P & {\tt stat.rank} & Rank or order in list of sorted value\\
P & {\tt stat.rms} & Root mean square  Square root of sum of squared values or quadratic mean\\
\sptablerule
\end{longtable}

\subsubsection*{Amendments/clarifications}
Definition for 
\begin{itemize}
\item {\tt phys.area} Area (in surface, not angular units)
\item {\tt stat.stdev} Standard deviation as the square root of the variance                                                           
\end{itemize}

\subsection{Changes from WD v1.23-20150608}
Text of Abstract, last two lines. Added reference to \citet{tn:solar-system-ucd}.

Section 1.1 Definition: ``Abbreviations are used in contexts where their meaning is unambiguous'' instead of ``kept to a minimum...''

\subsubsection*{Amendments/clarifications}
\begin{flushleft}
Description changed in words: 
{\tt em.UV.10-50nm}, {\tt em.UV.100-200nm}, {\tt em.UV.200-300nm}, {\tt meta.id.PI}, 
{\tt phot.flux}, {\tt phot.fluence}, {\tt src.class}.
\end{flushleft}

\subsubsection*{Additions}
\footnotesize\begin{longtable}[h!]{c|p{40ex}|p{0.45\textwidth}}
\sptablerule
Q & {\tt em.freq.cutoff} & Cutoff frequency\\
Q & {\tt em.freq.resonance} & Resonance frequency\\
S & {\tt em.pw} & Plasma waves (trapped in local medium)\\
S & {\tt em.radio.20MHz} & Radio below 20 MHz\\
Q & {\tt instr.experiment} & Experiment or group of instruments\\
Q & {\tt meta.calibLevel} & Processing/calibration level\\
S & {\tt meta.preview} & Related to a preview operation (for a dataset)\\
Q & {\tt meta.query} & Related to query posed to an information system or database\\
Q & {\tt meta.ref.ivoid} & An identifier following the IVOA Identifiers recommendation\\
S & {\tt obs.calib.dark} & Related to dark current calibration\\
S & {\tt obs.occult} & Observation of occultation phenomenon by solar system objects\\
S & {\tt obs.transit} & Observation of transit phenomenon: exo-planets\\
E & {\tt phot.radiance} & Radiance as energy flux per solid angle\\
S & {\tt phys.aerosol} & Relative to aerosol\\
Q & {\tt phys.density.phaseSpace} & Density in the phase space\\
S & {\tt phys.dust} & Relative to dust\\
E & {\tt phys.fluence} & Radiant photon energy received by a surface per unit area, or irradiance of a surface integrated over time of irradiation\\
Q & {\tt phys.flux} & Flux or flow of particle, energy, etc.\\
Q & {\tt phys.flux.energy} & Energy flux, heat flux\\
Q & {\tt phys.mass.inertiaMomentum} & Momentum of inertia or rotational inertia\\
S & {\tt phys.particle} & Related to physical particles\\
S & {\tt phys.particle.neutron} & Related to neutron\\
S & {\tt phys.particle.proton} & Related to proton\\
S & {\tt phys.particle.alpha} & Related to alpha particle\\
S & {\tt phys.phaseSpace} & Related to phase space\\
Q & {\tt phys.potential} & Potential (electric, gravitational, etc.)\\
Q & {\tt phys.size.smedAxis} & Linear semi median axis for 3D ellipsoids\\
Q & {\tt phys.volume} & Volume (in cubic units)\\
Q & {\tt pos.outline} & Set of points outlining a region (contour)\\
Q & {\tt src.orbital.Tisserand} & Tisserand parameter (generic)\\
Q & {\tt src.orbital.TissJ} & Tisserand parameter with respect to Jupiter\\
Q & {\tt time.period.revolution} & Period of revolution of a body around a primary one (similar to year)\\
Q & {\tt time.period.rotation} & Period of rotation of a body around its axis (similar to day)\\
\sptablerule
\end{longtable}

\subsubsection*{Deletions/replacements}
\begin{itemize}
\item deleted: {\tt em.UV.FUV} 
\item deleted: {\tt phys.mol.qn}; replaced by: {\tt phys.atmol.qn}
\item deleted: {\tt pos.bodyrc.long}; replaced by: {\tt pos.bodyrc.lon}
\item deleted: {\tt pos.eop.nutation}; replaced by: {\tt pos.nutation}
\end{itemize}

\subsubsection*{Deprecated}
\begin{itemize}
\item {\tt meta.ref.ivorn}: The term IVORN should not be used any more for IVOA Identifiers (IVOIDs). 
In UCDs, {\tt meta.ref.ivoid} should be used instead.
\end{itemize} 

\subsection{Changes from PR v1.22}

Text of pararagraph 1.1 (2), last three lines;

List of {\tt em} bands reordered according to wavelength/frequencies.

\subsubsection*{Amendments/clarifications}
Description changed in words: {\tt phys.atmol.qn}

\subsubsection*{Additions}
{\tt em.line.Hdelta}, {\tt em.line.Lyalpha}, {\tt em.line.CO}.

\subsubsection*{Deletions/replacements}
\begin{itemize}
\item deleted: {\tt phys.mol.qn}; replaced by: {\tt phys.atmol.qn}
\end{itemize}

\subsection{Changes from PR v1.21}
\subsubsection*{Amendments/clarifications}
\begin{itemize}
\item Syntax flag changed in words: {\tt phys.polarization}
\item \begin{flushleft}
Description changed in words: 
{\tt em.IR.FIR}, {\tt em.IR.MIR}, {\tt em.IR.NIR}, {\tt em.line.OIII}
\end{flushleft}
\end{itemize}

\subsection{Changes from PR v1.2}
\subsubsection*{Additions}
\begin{itemize}
\item {\tt spect.continuum}
\end{itemize}

\subsection{Changes from REC v1.11 (Rec20051231)}
\subsubsection*{Amendments/clarifications}
\begin{itemize}
\item Spelling: {\tt phys.atmol.sWeight}
\item Syntax flag changed in words: {\tt phys.atmol}, {\tt spect.line}
\item \begin{flushleft}
Description changed in words: 
{\tt meta.dataset}, {\tt obs.atmos}, {\tt phot.color.reddFree}, 
{\tt phys.size}, {\tt phys.size.diameter}, {\tt phys.size.radius}, {\tt stat.param}, {\tt stat.value}, 
{\tt time}, {\tt time.epoch}, {\tt time.interval}, {\tt time.period}, {\tt time.phase}.
\end{flushleft}
\end{itemize}

\subsubsection*{Additions}
\begin{flushleft}
{\tt em.bin}, {\tt em.binSize}, {\tt em.IR.FIR}, {\tt em.IR.MIR}, {\tt em.IR.NIR}, {\tt em.UV.FUV}, 
{\tt meta.abstract}, {\tt meta.code.status}, {\tt meta.email}, {\tt meta.id.PI}, {\tt meta.id.CoI}, 
{\tt meta.ref.ivorn}, {\tt meta.ref.uri}, {\tt obs.calib.flat}, {\tt obs.exposure}, {\tt obs.proposal}, 
{\tt obs.proposal.cycle}, {\tt obs.sequence}, {\tt phys.atmol.symmetry}, {\tt phys.atmol.sWeight.nuclear}, 
{\tt phys.cosmology}, {\tt phys.damping}, {\tt phys.entropy}, {\tt phys.particle.neutrino}, {\tt phys.virial}, 
{\tt spect.line.strength}, {\tt src.calib}, {\tt src.calib.guideStar}, {\tt src.net}, {\tt stat.filling}, 
{\tt stat.probability}, {\tt stat.uncalib}, {\tt time.creation}, {\tt time.duration}, {\tt time.end}, 
{\tt time.processing}, {\tt time.publiYear}, {\tt time.release}, {\tt time.star}
\end{flushleft}

\subsubsection*{Deletions/replacements}
\begin{itemize}
\item deleted: {\tt phys.atmol.damping}; replaced by: {\tt phys.damping} with description: Atomic damping quantities (van der Waals)
\item deleted: {\tt phys.atmol.qn.I}; replaced by: {\tt phys.atmol.qn} with description: Nuclear spin quantum number
\item deleted: {\tt time.event}; replaced by: {\tt time.duration} with description: Duration of an event or phenomenon
\item deleted: {\tt time.event.end}; replaced by: {\tt time.end} with description: End time of event or phenomenon
\item deleted: {\tt time.event.start}; replaced by: {\tt time.start} with description: Start time of event or phenomenon
\item deleted: {\tt time.expo}; replaced by: {\tt time.duration;obs.exposure} with description: Exposure on-time, duration
\item deleted: {\tt time.expo.end}; replaced by: {\tt time.end;obs.exposure} with description: End time of exposure
\item deleted: {\tt time.expo.start}; replaced by: {\tt time.start;obs.exposure} with description: Start time of exposure
\item deleted: {\tt time.obs}; replaced by: {\tt time.duration;obs} with description: Observation on-time, duration
\item deleted: {\tt time.obs.end}; replaced by: {\tt time.end;obs} with description: End time of observation
\item deleted: {\tt time.obs.start}; replaced by: {\tt time.start;obs} with description: Start time of observation
\end{itemize}

\subsection{Changes from v1.10}
\begin{enumerate}
\item A few minor changes to the text have been done
\item All UCD words are now compliant with the UCD recommendation. The corresponding changes are described below
\item The following words have been deprecated:

\begin{tabular}{|l|l|}
\sptablerule
Deprecated UCD & New corresponding UCD\\
\sptablerule
{\tt phot.fluxDens} & {\tt phot.flux.density}\\
{\tt phot.fluxDens.sb} & {\tt phot.flux.density.sb}\\ 
{\tt phys.at*} & {\tt phys.atmol*}\\
{\tt phys.atmol.coll} & {\tt phys.atmol.collisional}\\
{\tt phys.atmol.ion} & {\tt phys.atmol.ionStage}\\
{\tt phys.atmol.trans} & {\tt phys.atmol.transition}\\
{\tt phys.energyDensity} & {\tt phys.energy.density}\\
{\tt phys.massToLight} & {\tt phys.composition.massLightRatio}\\
{\tt phys.massYield} & {\tt phys.composition.yield}\\
{\tt spect.doppler} & {\tt spect.dopplerParam}\\
\sptablerule
\end{tabular}

\item The following word has been created: {\tt phys.composition}
\item The section Changes from previous versions has been reformatted
\end{enumerate}

\subsection{Changes from v1.0}
\begin{enumerate}
\item Descriptions have been changed for the following words: {\tt em.line}, {\tt instr.pixel}, 
{\tt phys.gravity}, {\tt pos.earth.altitude}
\item The syntax flags changed for words: {\tt instr.filter}, {\tt phys.angSize}
\item The following words have been deprecated:

\begin{tabular}{|l|l|}
\sptablerule
Deprecated UCD & New corresponding UCD\\
\sptablerule
{\tt instr.filter.transm} & {\tt phys.transm;instr.filter}\\
{\tt phys.mass.light} & {\tt phys.massToLight}\\
{\tt pos.resolution} & {\tt pos.angResolution}\\
{\tt pos.satellite} & {\tt pos.bodyrc}\\
\sptablerule
\end{tabular}

\item \begin{flushleft}
The following words have been created: 
{\tt phys.polarization.circular}, 
{\tt phys.polarization.linear}, {\tt phys.size.axisRatio}, {\tt pos.bodyrc.alt}, 
{\tt pos.bodyrc.lat}, {\tt pos.bodyrc.long}, {\tt time.event}, {\tt time.event.end}, 
{\tt time.event.start}.
\end{flushleft}
\end{enumerate}

\subsection{Changes from v1.01}

The following words have been restored to their previous spelling (v1.00): 
\begin{flushleft}
{\tt phot.fluDensity}, 
{\tt phys.energDensity}, {\tt phys.mYield}, {\tt phot.fluxDensity}, {\tt phys.energyDensity}, 
{\tt phys.massYield}.
\end{flushleft}

A note has been added to indicate that these words do not strictly comply with the UCD1+ Rec.

\subsection{Changes from v1.00}
\begin{enumerate}
\item \begin{flushleft}
Descriptions have been changed for the following words: 
{\tt em.IR.H}, {\tt em.IR.J}, 
{\tt em.IR.K}, {\tt em.X-ray.hard}, {\tt em.X-ray.medium}, {\tt em.X-ray.soft}, {\tt em.gamma.hard}, 
{\tt em.gamma.soft}, {\tt em.opt.B}, {\tt em.opt.I}, {\tt em.opt.R}, {\tt em.opt.U}, {\tt em.opt.V}, 
{\tt instr.bandpass}, {\tt phot.count}, {\tt phys.density}, {\tt phys.mol.dipole.electric}, 
{\tt phys.mol.dipole.magnetic}, {\tt phys.mol.quadrupole.electric}, {\tt pos.angDistance}, 
{\tt pos.precess}, {\tt src}, {\tt src.class.distance}, {\tt src.class.richness}, 
{\tt src.class.starGalaxy}, {\tt src.class.struct}, {\tt time.expo}, {\tt time.expo.end}, 
{\tt time.expo.start}, {\tt time.interval}
\end{flushleft}
\item The following words have been deprecated: 

\begin{longtable}{|l|l|}
\sptablerule
Deprecated UCD & New corresponding UCD\\
\sptablerule
{\tt instr.angRes} & {\tt pos.resolution}\\
{\tt instr.obsty.site} & {\tt pos.earth.altitude;instr.obsty}\\ 
{\tt instr.obsty.site.seeing} & {\tt instr.obsty.seeing}\\
{\tt instr.spect} & {\tt instr}\\
{\tt instr.spect.dispersion} & {\tt instr.dispersion}\\
{\tt instr.spect.order} & {\tt instr.order}\\
{\tt instr.spect.resolution} & {\tt spect.resolution}\\
{\tt instr.tel.focus} & {\tt instr.tel.focalLength}\\
{\tt meta.fits.software} & {\tt meta.software }\\
{\tt obs.air} & {\tt obs.atmos}\\
{\tt obs.air.extinction} & {\tt obs.atmos.extinction}\\
{\tt obs.air.mass} & {\tt obs.airMass}\\
{\tt phot.fluxDens} & {\tt phot.fluDens}\\
{\tt phot.fluxDens.sb} & {\tt phot.fluDens.sb}\\ 
{\tt phot.sb} & {\tt phot.mag.sb}\\
{\tt phys.at.branchingRatio} & {\tt phys.atmol.branchingRatio}\\
{\tt phys.at.crossSection} & {\tt phys.atmol.crossSection}\\
{\tt phys.at.lineShift} & {\tt phys.atmol.lineShift}\\
{\tt phys.at.moment} & \\
{\tt phys.at.moment.electric} & {\tt phys.at.radiationType}\\ 
{\tt phys.at.moment.magnetic} & {\tt phys.at.radiationType}\\
{\tt phys.at.qn.S} & {\tt phys.at.qn}\\
{\tt phys.at.qn.L} & {\tt phys.at.qn}\\
{\tt phys.at.qn.J} & {\tt phys.at.qn}\\
{\tt phys.at.qn.F} & {\tt phys.at.qn}\\
{\tt phys.atmol.state.final} & {\tt  phys.atmol.final}\\
{\tt phys.atmol.state.initial} & {\tt phys.atmol.initial}\\
{\tt phys.massYield} & {\tt phys.mYield}\\
{\tt phys.mol.quadrupole.magnetic} & {\tt phys.at.radiationType}\\
{\tt phys.refraction} & {\tt phys.refractIndex}\\
{\tt pos.az.ha} & {\tt pos.eq.ha}\\
{\tt pos.earth.nutation} & {\tt pos.eop.nutation}\\
{\tt spect.veloc} & {\tt spect.dopplerVeloc}\\
{\tt src.fwhm} & {\tt phys.angSize;src}\\
{\tt src.orbital.veloc} & {\tt phys.veloc.orbital}\\
{\tt src.veloc} & {\tt phys.veloc}\\
{\tt src.veloc.ang} & {\tt phys.veloc.ang}\\
{\tt src.veloc.cmb} & {\tt phys.veloc;pos.cmb}\\ 
{\tt src.veloc.dispersion} & {\tt phys.veloc.dispersion}\\
{\tt src.veloc.escape} & {\tt phys.veloc.escape}\\
{\tt src.veloc.expansion} & {\tt phys.veloc.expansion}\\ 
{\tt src.veloc.lg} & {\tt phys.veloc;pos.lg}\\
{\tt src.veloc.lsr} & {\tt phys.veloc;pos.lsrv}\\
{\tt src.veloc.microTurb} & {\tt phys.veloc.microTurb}\\
{\tt src.veloc.pulsat} & {\tt phys.veloc.pulsat}\\
{\tt src.veloc.rotat} & {\tt phys.veloc.rotat}\\
\sptablerule
\end{longtable}

\item \begin{flushleft}
The syntax flags changed for words: 
{\tt instr.fov}, {\tt instr.obsty}, {\tt meta.file}, 
{\tt phys.angSize}, {\tt pos.cartesian}, {\tt stat.fit.omc}
\end{flushleft}
\item \begin{flushleft}
The following words have been created: 
{\tt instr.dispersion}, {\tt instr.order}, {\tt 
instr.tel.focalLength}, {\tt meta.curation}, {\tt meta.software}, {\tt meta.version}, {\tt 
obs.atmos}, {\tt obs.atmos.extinction}, {\tt obs.airMass}, {\tt obs.atmos.refractAngle}, 
{\tt obs.calib}, {\tt phys.at.radiationType}, {\tt phys.atmol.branchingRatio}, {\tt 
phys.atmol.crossSection}, {\tt phys.atmol.lifetime}, {\tt phys.atmol.lineShift}, {\tt 
phys.energDensity}, {\tt phys.refractIndex}, {\tt phys.transmission}, {\tt pos.eq.ha}, 
{\tt pos.az.azi}, {\tt pos.bodyrc}, {\tt pos.cmb}, {\tt pos.earth.altitude}, {\tt pos.eop}, 
{\tt pos.eop.nutation}, {\tt pos.lg}, {\tt pos.lsr}, {\tt pos.phaseAng}, {\tt pos.resolution}, 
{\tt spect.resolution}, {\tt spect.dopplerVeloc}, {\tt spect.dopplerVeloc.radio}, {\tt 
spect.dopplerVeloc.opt}, {\tt src.orbital.meanMotion}, {\tt phys.veloc}, {\tt phys.veloc.ang}, 
{\tt phys.veloc.dispersion}, {\tt phys.veloc.escape}, {\tt phys.veloc.expansion}, {\tt 
phys.veloc.microTurb}, {\tt phys.veloc.orbital}, {\tt phys.veloc.pulsat}, {\tt phys.veloc.rotat}, 
{\tt phys.veloc.transverse}, {\tt time.obs}, {\tt time.obs.end}, {\tt time.obs.start}.
\end{flushleft}
\end{enumerate}

\subsection{Changes from v0.2}
\begin{enumerate}
\item Section 1.2 has been simplified
\item new syntax codes (E, C, V) have been introduced, and described in appendix A
\item The following words have been renamed

\begin{longtable}{|l|l|}
\sptablerule
Deprecated UCD & New corresponding UCD\\
\sptablerule
{\tt em.line.21cm} & {\tt em.line.HI}\\
{\tt instr.ang-res} & {\tt instr.angRes}\\
{\tt instr.sky-level} & {\tt instr.skyLevel}\\
{\tt instr.sky-temp} & {\tt instr.skyTemp}\\
{\tt instr.antenna-temp} & {\tt phot.antennaTemp}\\
{\tt phys.absorption.gf} & {\tt phys.gauntFactor}\\
{\tt phys.at.einstein} & {\tt phys.at.transProb}\\
{\tt phys.at.level} & {\tt phys.atmol.level}\\
{\tt phys.dispMeas} & {\tt phys.dispMeasure}\\ 
{\tt phys.distance} & {\tt pos.distance}\\
{\tt phys.polarization.rotMeas} & {\tt phys.polarization.rotMeasure}\\
{\tt phys.size.area} & {\tt phys.area}\\
{\tt pos.ang.separation} & {\tt pos.angDistance}\\ 
{\tt pos.ec} & {\tt pos.ecliptic}\\
{\tt pos.ec.lat} & {\tt pos.ecliptic.lat}\\
{\tt pos.ec.lon} & {\tt pos.ecliptic.lon}\\
{\tt pos.ee} & {\tt pos.errorEllipse}\\ 
{\tt pos.gal} & {\tt pos.galactic}\\
{\tt pos.gal.lat} & {\tt pos.galactic.lat}\\
{\tt pos.gal.lon} & {\tt pos.galactic.lon}\\
{\tt pos.sg} & {\tt pos.supergalactic}\\
{\tt pos.sg.lat} & {\tt pos.supergalactic.lat}\\
{\tt pos.sg.lon} & {\tt pos.supergalactic.lon}\\
{\tt src.class.star-galaxy} & {\tt src.class.starGalaxy}\\
\sptablerule
\end{longtable}

\item \begin{flushleft}
The following words have been created: 
{\tt instr.beam}, {\tt meta.code.error}, {\tt meta.id.part}, {\tt phot.flux.sb}, 
{\tt phys.angArea}, {\tt phys.angSize}, {\tt phys.angSize.smajAxis}, {\tt phys.angSize.sminAxis}, {\tt phys.area}, 
{\tt phys.at.damping}, {\tt phys.at.weight}, {\tt phys.atmol.excitation}, {\tt phys.mol.dissociation}, {\tt 
phys.recombination.coeff}, {\tt phys.size.smajAxis}, {\tt phys.size.sminAxis}, {\tt pos.cartesian}, {\tt pos.cartesian.x},
{\tt pos.cartesian.y}, {\tt pos.cartesian.z}, {\tt pos.distance}, {\tt pos.eq.spd}, {\tt pos.galactocentric}, {\tt 
pos.geocentric}, {\tt pos.healpix}, {\tt pos.heliocentric}, {\tt pos.HTM}, {\tt pos.lambert}, {\tt pos.satellite}, 
{\tt spect.line.broad.Stark}, {\tt spect.veloc}, {\tt src.redshift.phot}, {\tt stat.correlation}, {\tt time.lifetime}.
\end{flushleft}
\item Some words have been removed. The following table summarizes, when relevant, the suggested replacement to be used. 

\begin{longtable}{|l|l|}
\sptablerule
Deprecated UCD & New corresponding UCD\\
\sptablerule
{\tt instr.area} & {\tt phys.area;instr}\\
{\tt instr.beam-width} & {\tt phys.angSize;instr.beam}\\
{\tt meta.table.axis} & {\tt phys.size;meta.table}\\
{\tt phot.color.Cous} & {\tt phot.color}\\
{\tt phot.color.Gen} & {\tt phot.color}\\
{\tt phot.color.Gunn} & {\tt phot.color}\\
{\tt phot.color.JHN} & {\tt phot.color}\\
{\tt phot.color.STR} & {\tt phot.color}\\
{\tt phot.color.STR.c1} & {\tt phot.color}\\ 
{\tt phot.color.STR.b-y} & {\tt phot.color}\\
{\tt phot.color.STR.m1} & {\tt phot.color}\\
{\tt phys.at.lineBroad} & {\tt spect.line.broad}\\ 
{\tt phys.distance.compon} & {\tt pos.distance;pos.cartesian.x} (or {\tt y}, {\tt z})\\
{\tt phys.distance.gc} & {\tt pos.distance;pos.galactocentric}\\
{\tt phys.electron.energy} & {\tt phys.energy;phys.electron}\\
{\tt phys.extension} & {\tt phys.angSize or phys.size}\\
{\tt phys.mass.fraction} & {\tt phys.mass;arith.ratio}\\
{\tt phys.polarization.posAng} & {\tt pos.posAng;phys.polarization}\\
{\tt pos.ang} & \\
{\tt pos.det} & {\tt pos.cartesian;instr.det}\\
{\tt pos.eq.dec.arcsec} & \\
{\tt pos.eq.ra.minutes} & \\
{\tt pos.eq.ra.seconds} & \\
{\tt pos.gal.compon} & {\tt pos.cartesian;pos.galactic}\\
{\tt pos.pm.dec} & {\tt pos.pm;pos.eq.dec}\\
{\tt pos.pm.ra} & {\tt pos.pm;pos.eq.ra}\\
{\tt pos.precess.dec} & {\tt pos.precess;pos.eq.dec}\\
{\tt pos.precess.ra} & {\tt pos.precess;pos.eq.ra}\\
{\tt pos.proj} & \\
{\tt pos.sg.compon} & {\tt pos.cartesian;pos.supergalactic}\\
{\tt src.orbital.energy} & {\tt phys.energy;src.orbital}\\
{\tt src.orbital.separation} & {\tt pos.angDistance;src.orbital}\\
{\tt src.orbital.size} & {\tt phys.size;src.orbital}\\
{\tt src.separation} & {\tt pos.angDistance;src}\\
{\tt src.veloc.compon} & {\tt src.veloc;pos.cartesian}\\ 
{\tt src.veloc.gc} & {\tt src.veloc;pos.galactocentric}\\
{\tt src.veloc.geoc} & {\tt src.veloc;pos.geocentric}\\
{\tt src.veloc.hc} & {\tt src.veloc;pos.heliocentric}\\
\sptablerule
\end{longtable}
\end{enumerate}

\subsection{Changes from v0.1}
\begin{enumerate}
\item Descriptions of the words were improved.
\item Designation of commonly used lines have been moved to {\tt em.line.*}. As a consequence, terms like 
{\tt em.IR.K.Brgamma} or {\tt spect.index.Hbeta} have been removed.
\item {\tt phys.at} and {\tt phys.mol} have been completely reorganized to improve the overall description 
of this domain. A new branch {\tt phys.atmol} has been introduced to group concepts shared between {\tt 
phys.at} and {\tt phys.mol}.
\item The {\tt phot.color} section was significantly simplified.
\item Missing nodes of the tree were added (e.g. {\tt em.gamma}, {\tt em.mm}, {\tt pos.sg}).
\item Creation of new words: {\tt em.wavenumber}, {\tt meta.ucd}, {\tt stat.error.sys}.
\item Typos were corrected in {\tt em.opt.*} units and a few other descriptions.
\end{enumerate}

\bibliography{ivoatex/ivoabib,ivoatex/docrepo,localrefs}
\end{document}


