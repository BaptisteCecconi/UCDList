\documentclass[11pt,a4paper]{ivoa}
\input tthdefs

% let's make it a bit easier on us by widening
% the paper a bit for the humongous tables.
\usepackage[width=16cm,height=24cm]{geometry} 
\usepackage{hyperref}
\usepackage{verbatim}
\usepackage{longtable}
\usepackage{array}
\usepackage{natbib} 
\usepackage{enumitem} 
\title{UCD1+ controlled vocabulary\\ \emph{Updated List of Terms}}

% having descriptions in a narrow column is painful -- the following
% is an attempt to define a suitable column style ("D"escription)
\newcolumntype{D}[1]{>{\raggedright\tolerance=7000\let\newline\\\arraybackslash
  \hspace{0pt}}p{#1}}

\iftth
\def\ucd{\begingroup\tt}
\def\enducd{\endgroup}
\else
\begingroup
\gdef\breakabledot{\discretionary{...}{...}{.}}
\catcode`\.=\active
\gdef\ucd{\begingroup\tt\catcode`\.=\active\let.=\breakabledot}
\gdef\enducd{\endgroup}
\endgroup
\fi

% having descriptions in a narrow column is painful -- the following
% is an attempt to define a suitable column style ("D"escription)
\newcolumntype{D}[1]{>{\raggedright\tolerance=7000\let\newline\\\arraybackslash
  \hspace{0pt}}p{#1}}

\iftth
\def\ucd{\begingroup\tt}
\def\enducd{\endgroup}
\else
\begingroup
\gdef\breakabledot{\discretionary{...}{...}{.}}
\catcode`\.=\active
\gdef\ucd{\begingroup\tt\catcode`\.=\active\let.=\breakabledot}
\gdef\enducd{\endgroup}
\endgroup
\fi

% see ivoatexDoc for what group names to use here
\ivoagroup{Semantics}

\author{Andrea Preite Martinez}
\author{Mireille Louys}
\author{Baptiste Cecconi}
\author{S\'ebastien Derri\`ere}
\author{Fran\c cois Ochsenbein}
\author{St\'ephane Erard}

\editor[mailto:baptiste.cecconi@obspm.fr]{Baptiste Cecconi, Mireille Louys}

\previousversion[http://www.ivoa.net/documents/UCD1+/20180527/REC-UCDlist-1.3-20180527.pdf]{The UCD1+ 
controlled vocabulary 1.3}

\setcounter{secnumdepth}{5}  

\begin{document}
\begin{abstract}
This document describes the \emph{list of controlled terms}  building up the corpus of the Unified Content Descriptors, Version 1+ (UCD1+). 

The document describing the UCD1+ principles can be found at the URL: \url{https://ivoa.net/documents/UCD1+/20180527/index.html}. 
It is an IVOA Recommendation. 
The process to maintain and enrich the UCD list of terms is standardized in  \url{https://ivoa.net/documents/UCDlistMaintenance/}.
It states that successive versions of the UCD1+ vocabulary are distributed in Endorsed Notes within the IVOA.
This version is the first Endorsed Note for UCD list, currently proposed. 
It contains new UCD words discussed and accepted by the Semantics Working Group during the UCD list v1.3 RFM (request for modification) available at \url{https://wiki.ivoa.net/twiki/bin/view/IVOA/UCDList_1-3_RFM}.

\end{abstract} 

\section{Introduction}

A UCD is a string which contains textual tokens called `words', separated by semicolons(;). A word is composed of 
`atoms', separated by periods(.). The hierarchy is as follows: 
$$
\textrm{atoms} \rightarrow \textrm{words} \rightarrow \textrm{composed words}
$$
UCD1+ are either single words, or a composition of several words.

UCDs are ``controlled'' through a process defined in the IVOA. See \citet{2005ivoa.spec.0819D} and section \ref{sec:words} below. 
Control is exercised at the level of words (UCD1+) and at the level of the vocabulary (atoms) used to form words. A 
consistent list of atoms will be maintained, making sure that the same atom always means the same thing, even if used 
in combination with different other atoms. 


\section{List of valid words}
\label{sec:list}

All words are preceded by a `syntax' code that can help in the process of building composed UCD1+.
\begin{enumerate}
\item ``P'' means that the word can only be used as ``primary'' or first word;
\item ``S'' stands for only secondary: the word cannot be used as the first word to describe a 
single quantity;
\item ``Q'' means that the word can be used indifferently as first or secondary word;

The following cases behave as Q prefix and can be combined  as primary or secondary.
They specialize the combination rules:  
	\begin{enumerate}[label*=\arabic*.]
	\item ``E'' means a photometric quantity, and can be followed by a word describing a part of 
	the electromagnetic spectrum;
	\item ``C'' is a colour index, and can be followed by two successive word describing a part of 
	the electromagnetic spectrum;
	\item ``V'' stands for vector. Such a word can be followed by another describing the axis or 
	reference frame in which the measurement is done.
	\end{enumerate}
\end{enumerate}

For typographic reasons some long UCD atoms are printed on two
lines in the following table.  In these cases, \texttt{some.long...
...ucd.x} is to be read as \texttt{some.long.ucd.x}.

\newpage

%\begin{table}[h!]
%\begin{tabular}{c|l|p{0.5\textwidth}}
\footnotesize\begin{longtable}[h!]{c|D{0.23\textwidth}|D{0.67\textwidth}}
\sptablerule
\multicolumn{2}{l|}{\textbf{UCD word}}&\textbf{Description}\\
\sptablerule
% GENERATED: python3 convert_ucd_list_to_tex.py ucd-list-1_4.txt
Q & \ucd arith \enducd& Arithmetic quantities\\
S & \ucd arith.diff \enducd& Difference between two quantities described by the same UCD\\
P & \ucd arith.factor \enducd& Numerical factor\\
P & \ucd arith.grad \enducd& Gradient\\
P & \ucd arith.rate \enducd& Rate (per time unit)\\
S & \ucd arith.ratio \enducd& Ratio between two quantities described by the same UCD\\
S & \ucd arith.squared \enducd& Squared quantity\\
S & \ucd arith.sum \enducd& Summed or integrated quantity\\
S & \ucd arith.variation \enducd& Generic variation of a quantity\\
Q & \ucd arith.zp \enducd& Zero point\\
S & \ucd em \enducd& Electromagnetic spectrum\\
S & \ucd em.IR \enducd& Infrared part of the spectrum\\
S & \ucd em.IR.J \enducd& Infrared between 1.0 and 1.5 micron\\
S & \ucd em.IR.H \enducd& Infrared between 1.5 and 2 micron\\
S & \ucd em.IR.K \enducd& Infrared between 2 and 3 micron\\
S & \ucd em.IR.3-4um \enducd& Infrared between 3 and 4 micron\\
S & \ucd em.IR.4-8um \enducd& Infrared between 4 and 8 micron\\
S & \ucd em.IR.8-15um \enducd& Infrared between 8 and 15 micron\\
S & \ucd em.IR.15-30um \enducd& Infrared between 15 and 30 micron\\
S & \ucd em.IR.30-60um \enducd& Infrared between 30 and 60 micron\\
S & \ucd em.IR.60-100um \enducd& Infrared between 60 and 100 micron\\
S & \ucd em.IR.NIR \enducd& Near-Infrared, 1-5 microns\\
S & \ucd em.IR.MIR \enducd& Medium-Infrared, 5-30 microns\\
S & \ucd em.IR.FIR \enducd& Far-Infrared, 30-100 microns\\
S & \ucd em.UV \enducd& Ultraviolet part of the spectrum\\
S & \ucd em.UV.10-50nm \enducd& Ultraviolet between 10 and 50 nm EUV extreme UV\\
S & \ucd em.UV.50-100nm \enducd& Ultraviolet between 50 and 100 nm\\
S & \ucd em.UV.100-200nm \enducd& Ultraviolet between 100 and 200 nm FUV Far UV\\
S & \ucd em.UV.200-300nm \enducd& Ultraviolet between 200 and 300 nm NUV near UV\\
S & \ucd em.X-ray \enducd& X-ray part of the spectrum\\
S & \ucd em.X-ray.soft \enducd& Soft X-ray (0.12 - 2 keV)\\
S & \ucd em.X-ray.medium \enducd& Medium X-ray (2 - 12 keV)\\
S & \ucd em.X-ray.hard \enducd& Hard X-ray (12 - 120 keV)\\
Q & \ucd em.bin \enducd& Channel / instrumental spectral bin coordinate (bin number)\\
Q & \ucd em.energy \enducd& Energy value in the em frame\\
Q & \ucd em.freq \enducd& Frequency value in the em frame\\
Q & \ucd em.freq.cutoff \enducd& cutoff frequency\\
Q & \ucd em.freq.resonance \enducd& resonance frequency\\
S & \ucd em.gamma \enducd& Gamma rays part of the spectrum\\
S & \ucd em.gamma.soft \enducd& Soft gamma ray (120 - 500 keV)\\
S & \ucd em.gamma.hard \enducd& Hard gamma ray (>500 keV)\\
S & \ucd em.line \enducd& Designation of major atomic lines\\
S & \ucd em.line.HI \enducd& 21cm hydrogen line\\
S & \ucd em.line.Lyalpha \enducd& H-Lyalpha line\\
S & \ucd em.line.Halpha \enducd& H-alpha line\\
S & \ucd em.line.Hbeta \enducd& H-beta line\\
S & \ucd em.line.Hgamma \enducd& H-gamma line\\
S & \ucd em.line.Hdelta \enducd& H-delta line\\
S & \ucd em.line.Brgamma \enducd& Bracket gamma line\\
S & \ucd em.line.OIII \enducd& [OIII] line whose rest wl is 500.7 nm\\
S & \ucd em.line.CO \enducd& CO radio line,  e.g 12CO(1-0) at 115GHz\\
S & \ucd em.mm \enducd& Millimetric/submillimetric part of the spectrum\\
S & \ucd em.mm.30-50GHz \enducd& Millimetric between 30 and 50 GHz\\
S & \ucd em.mm.50-100GHz \enducd& Millimetric between 50 and 100 GHz\\
S & \ucd em.mm.100-200GHz \enducd& Millimetric between 100 and 200 GHz\\
S & \ucd em.mm.200-400GHz \enducd& Millimetric between 200 and 400 GHz\\
S & \ucd em.mm.400-750GHz \enducd& Millimetric between 400 and 750 GHz\\
S & \ucd em.mm.750-1500GHz \enducd& Millimetric between 750 and 1500 GHz\\
S & \ucd em.mm.1500-3000GHz \enducd& Millimetric between 1500 and 3000 GHz\\
S & \ucd em.opt \enducd& Optical part of the spectrum\\
S & \ucd em.opt.U \enducd& Optical band between 300 and 400 nm\\
S & \ucd em.opt.B \enducd& Optical band between 400 and 500 nm\\
S & \ucd em.opt.V \enducd& Optical band between 500 and 600 nm\\
S & \ucd em.opt.R \enducd& Optical band between 600 and 750 nm\\
S & \ucd em.opt.I \enducd& Optical band between 750 and 1000 nm\\
S & \ucd em.pw \enducd& Plasma waves (trapped in local medium)\\
S & \ucd em.radio \enducd& Radio part of the spectrum\\
S & \ucd em.radio.20MHz \enducd& Radio below 20 MHz\\
S & \ucd em.radio.20-100MHz \enducd& Radio between 20 and 100 MHz\\
S & \ucd em.radio.100-200MHz \enducd& Radio between 100 and 200 MHz\\
S & \ucd em.radio.200-400MHz \enducd& Radio between 200 and 400 MHz\\
S & \ucd em.radio.400-750MHz \enducd& Radio between 400 and 750 MHz\\
S & \ucd em.radio.750-1500MHz \enducd& Radio between 750 and 1500 MHz\\
S & \ucd em.radio.1500-3000MHz \enducd& Radio between 1500 and 3000 MHz\\
S & \ucd em.radio.3-6GHz \enducd& Radio between 3 and 6 GHz\\
S & \ucd em.radio.6-12GHz \enducd& Radio between 6 and 12 GHz\\
S & \ucd em.radio.12-30GHz \enducd& Radio between 12 and 30 GHz\\
Q & \ucd em.wavenumber \enducd& Wavenumber value in the em frame\\
Q & \ucd em.wl \enducd& Wavelength value in the em frame\\
Q & \ucd em.wl.central \enducd& Central wavelength\\
Q & \ucd em.wl.effective \enducd& Effective wavelength\\
Q & \ucd instr \enducd& Instrument\\
E & \ucd instr.background \enducd& Instrumental background\\
Q & \ucd instr.bandpass \enducd& Bandpass (e.g.: band name) of instrument\\
Q & \ucd instr.bandwidth \enducd& Bandwidth of the instrument\\
Q & \ucd instr.baseline \enducd& Baseline for interferometry\\
S & \ucd instr.beam \enducd& Beam\\
Q & \ucd instr.calib \enducd& Calibration parameter\\
S & \ucd instr.det \enducd& Detector\\
Q & \ucd instr.det.noise \enducd& Instrument noise\\
Q & \ucd instr.det.psf \enducd& Point Spread Function\\
Q & \ucd instr.det.qe \enducd& Quantum efficiency\\
Q & \ucd instr.dispersion \enducd& Dispersion of a spectrograph\\
Q & \ucd instr.experiment \enducd& Experiment or group of instruments\\
S & \ucd instr.filter \enducd& Filter\\
S & \ucd instr.fov \enducd& Field of view\\
S & \ucd instr.obsty \enducd& Observatory, satellite, mission\\
Q & \ucd instr.obsty.seeing \enducd& Seeing\\
Q & \ucd instr.offset \enducd& Offset angle respect to main direction of observation\\
Q & \ucd instr.order \enducd& Spectral order in a spectrograph\\
Q & \ucd instr.param \enducd& Various instrumental parameters\\
S & \ucd instr.pixel \enducd& Pixel (default size: angular)\\
S & \ucd instr.plate \enducd& Photographic plate\\
Q & \ucd instr.plate.emulsion \enducd& Plate emulsion\\
Q & \ucd instr.precision \enducd& Instrument precision\\
Q & \ucd instr.rmsf \enducd& Rotation Measure Spread Function\\
Q & \ucd instr.saturation \enducd& Instrument saturation threshold\\
Q & \ucd instr.scale \enducd& Instrument scale (for CCD, plate, image)\\
Q & \ucd instr.sensitivity \enducd& Instrument sensitivity, detection threshold\\
Q & \ucd instr.setup \enducd& Instrument configuration or setup\\
Q & \ucd instr.skyLevel \enducd& Sky level\\
Q & \ucd instr.skyTemp \enducd& Sky temperature\\
Q & \ucd instr.tel \enducd& Telescope\\
Q & \ucd instr.tel.focalLength \enducd& Telescope focal length\\
S & \ucd instr.voxel \enducd& Related to a voxel (n-D volume element with n>2)\\
P & \ucd meta \enducd& Metadata\\
P & \ucd meta.abstract \enducd& Abstract (of paper, proposal, etc.)\\
P & \ucd meta.bib \enducd& Bibliographic reference\\
P & \ucd meta.bib.author \enducd& Author name\\
P & \ucd meta.bib.bibcode \enducd& Bibcode\\
P & \ucd meta.bib.fig \enducd& Figure in a paper\\
P & \ucd meta.bib.journal \enducd& Journal name\\
P & \ucd meta.bib.page \enducd& Page number\\
P & \ucd meta.bib.volume \enducd& Volume number\\
Q & \ucd meta.calibLevel \enducd& Processing/calibration level\\
Q & \ucd meta.checksum \enducd& Numerical signature of digital data\\
P & \ucd meta.code \enducd& Code or flag\\
P & \ucd meta.code.class \enducd& Classification code\\
P & \ucd meta.code.error \enducd& Limit uncertainty error flag\\
P & \ucd meta.code.member \enducd& Membership code\\
P & \ucd meta.code.mime \enducd& MIME type\\
P & \ucd meta.code.multip \enducd& Multiplicity or binarity flag\\
P & \ucd meta.code.qual \enducd& Quality, precision, reliability flag or code\\
P & \ucd meta.code.status \enducd& Status code (e.g.: status of a proposal/observation)\\
P & \ucd meta.cryptic \enducd& Unknown or impossible to understand quantity\\
P & \ucd meta.curation \enducd& Identity of man/organization responsible for the data\\
Q & \ucd meta.dataset \enducd& Dataset\\
Q & \ucd meta.email \enducd& Curation/contact e-mail\\
S & \ucd meta.file \enducd& File\\
S & \ucd meta.fits \enducd& FITS standard\\
P & \ucd meta.id \enducd& Identifier, name or designation\\
P & \ucd meta.id.assoc \enducd& Identifier of associated counterpart\\
P & \ucd meta.id.CoI \enducd& Name of Co-Investigator\\
P & \ucd meta.id.cross \enducd& Cross identification\\
P & \ucd meta.id.parent \enducd& Identification of parent source\\
P & \ucd meta.id.part \enducd& Part of identifier, suffix or sub-component\\
P & \ucd meta.id.PI \enducd& Name of Principal Investigator or Co-PI\\
S & \ucd meta.main \enducd& Main value of something\\
S & \ucd meta.modelled \enducd& Quantity was produced by a model\\
P & \ucd meta.note \enducd& Note or remark (longer than a code or flag)\\
P & \ucd meta.number \enducd& Number (of things; e.g. nb of object in an image)\\
S & \ucd meta.preview \enducd& Related to a preview operation for a dataset\\
Q & \ucd meta.query \enducd& A query posed to an information system or database or a property of it\\
P & \ucd meta.record \enducd& Record number\\
P & \ucd meta.ref \enducd& Reference or origin\\
P & \ucd meta.ref.doi \enducd& DOI identifier (dereferenceable)\\
Q & \ucd meta.ref.ivoid \enducd& Identifier as recommended  in the IVOA  (dereferenceable)\\
P & \ucd meta.ref.ivorn \enducd& Identifier defined as IVORN, VO Resource Name (ivo://)  (deprecated)\\
P & \ucd meta.ref.uri \enducd& URI, universal resource identifier\\
P & \ucd meta.ref.url \enducd& URL, web address\\
S & \ucd meta.software \enducd& Software used in generating data\\
S & \ucd meta.table \enducd& Table or catalogue\\
P & \ucd meta.title \enducd& Title or explanation\\
Q & \ucd meta.ucd \enducd& UCD\\
P & \ucd meta.unit \enducd& Unit\\
P & \ucd meta.version \enducd& Version\\
S & \ucd obs \enducd& Observation\\
Q & \ucd obs.airMass \enducd& Airmass\\
S & \ucd obs.atmos \enducd& Atmosphere, atmospheric phenomena affecting an observation\\
Q & \ucd obs.atmos.extinction \enducd& Atmospheric extinction\\
Q & \ucd obs.atmos.refractAngle \enducd& Atmospheric refraction angle\\
S & \ucd obs.calib \enducd& Calibration observation\\
S & \ucd obs.calib.flat \enducd& Related to flat-field calibration observation (dome, sky, ..)\\
S & \ucd obs.calib.dark \enducd& Related to dark current calibration\\
S & \ucd obs.exposure \enducd& Exposure\\
S & \ucd obs.field \enducd& Region covered by the observation\\
S & \ucd obs.image \enducd& Image\\
Q & \ucd obs.observer \enducd& Observer, discoverer\\
S & \ucd obs.occult \enducd& Observation of occultation phenomenon by solar system objects\\
S & \ucd obs.transit \enducd& Observation of transit phenomenon  : exo-planets\\
Q & \ucd obs.param \enducd& Various observation or reduction parameter\\
S & \ucd obs.proposal \enducd& Observation proposal\\
Q & \ucd obs.proposal.cycle \enducd& Proposal cycle\\
S & \ucd obs.sequence \enducd& Sequence of observations, exposures or events\\
E & \ucd phot \enducd& Photometry\\
E & \ucd phot.antennaTemp \enducd& Antenna temperature\\
Q & \ucd phot.calib \enducd& Photometric calibration\\
C & \ucd phot.color \enducd& Color index or magnitude difference\\
Q & \ucd phot.color.excess \enducd& Color excess\\
Q & \ucd phot.color.reddFree \enducd& Dereddened color\\
E & \ucd phot.count \enducd& Flux expressed in counts\\
E & \ucd phot.fluence \enducd& Radiant photon energy received by a surface per unit area or irradiance of a surface integrated over time of irradiation\\
E & \ucd phot.flux \enducd& Photon flux or irradiance\\
Q & \ucd phot.flux.bol \enducd& Bolometric flux\\
E & \ucd phot.flux.density \enducd& Flux density (per wl/freq/energy interval)\\
E & \ucd phot.flux.density.sb \enducd& Flux density surface brightness\\
E & \ucd phot.flux.sb \enducd& Flux surface brightness\\
E & \ucd phot.limbDark \enducd& Limb-darkening coefficients\\
E & \ucd phot.mag \enducd& Photometric magnitude\\
E & \ucd phot.mag.bc \enducd& Bolometric correction\\
Q & \ucd phot.mag.bol \enducd& Bolometric magnitude\\
Q & \ucd phot.mag.distMod \enducd& Distance modulus\\
E & \ucd phot.mag.reddFree \enducd& Dereddened magnitude\\
E & \ucd phot.mag.sb \enducd& Surface brightness in magnitude units\\
E & \ucd phot.radiance \enducd& Radiance as energy flux per solid angle\\
Q & \ucd phys \enducd& Physical quantities\\
Q & \ucd phys.SFR \enducd& Star formation rate\\
E & \ucd phys.absorption \enducd& Extinction or absorption along the line of sight\\
Q & \ucd phys.absorption.coeff \enducd& Absorption coefficient (e.g. in a spectral line)\\
Q & \ucd phys.absorption.gal \enducd& Galactic extinction\\
Q & \ucd phys.absorption.opticalDepth \enducd& Optical depth\\
Q & \ucd phys.abund \enducd& Abundance\\
Q & \ucd phys.abund.Fe \enducd& Fe/H abundance\\
Q & \ucd phys.abund.X \enducd& Hydrogen abundance\\
Q & \ucd phys.abund.Y \enducd& Helium abundance\\
Q & \ucd phys.abund.Z \enducd& Metallicity abundance\\
Q & \ucd phys.acceleration \enducd& Acceleration\\
S & \ucd phys.aerosol \enducd& Relative to aerosol\\
Q & \ucd phys.albedo \enducd& Albedo or reflectance\\
Q & \ucd phys.angArea \enducd& Angular area\\
Q & \ucd phys.angMomentum \enducd& Angular momentum\\
E & \ucd phys.angSize \enducd& Angular size width diameter dimension extension major minor axis extraction radius\\
E & \ucd phys.angSize.smajAxis \enducd& Angular size extent or extension of semi-major axis\\
E & \ucd phys.angSize.sminAxis \enducd& Angular size extent or extension of semi-minor axis\\
Q & \ucd phys.area \enducd& Area (in surface, not angular units)\\
S & \ucd phys.atmol \enducd& Atomic and molecular physics (shared properties)\\
Q & \ucd phys.atmol.branchingRatio \enducd& Branching ratio\\
S & \ucd phys.atmol.collisional \enducd& Related to collisions\\
Q & \ucd phys.atmol.collStrength \enducd& Collisional strength\\
Q & \ucd phys.atmol.configuration \enducd& Configuration\\
Q & \ucd phys.atmol.crossSection \enducd& Atomic / molecular cross-section\\
Q & \ucd phys.atmol.element \enducd& Element\\
Q & \ucd phys.atmol.excitation \enducd& Atomic molecular excitation parameter\\
Q & \ucd phys.atmol.final \enducd& Quantity refers to atomic/molecular final/ground state, level, etc.\\
Q & \ucd phys.atmol.initial \enducd& Quantity refers to atomic/molecular initial state, level, etc.\\
Q & \ucd phys.atmol.ionStage \enducd& Ion, ionization stage\\
S & \ucd phys.atmol.ionization \enducd& Related to ionization\\
Q & \ucd phys.atmol.lande \enducd& Lande factor\\
S & \ucd phys.atmol.level \enducd& Atomic level\\
Q & \ucd phys.atmol.lifetime \enducd& Lifetime of a level\\
Q & \ucd phys.atmol.lineShift \enducd& Line shifting coefficient\\
Q & \ucd phys.atmol.number \enducd& Atomic number Z\\
Q & \ucd phys.atmol.oscStrength \enducd& Oscillator strength\\
Q & \ucd phys.atmol.parity \enducd& Parity\\
Q & \ucd phys.atmol.qn \enducd& Quantum number\\
Q & \ucd phys.atmol.radiationType \enducd& Type of radiation characterizing atomic lines (electric dipole/quadrupole, magnetic dipole)\\
Q & \ucd phys.atmol.symmetry \enducd& Type of nuclear spin symmetry\\
Q & \ucd phys.atmol.sWeight \enducd& Statistical weight\\
Q & \ucd phys.atmol.sWeight.nuclear \enducd& Statistical weight for nuclear spin states\\
Q & \ucd phys.atmol.term \enducd& Atomic term\\
S & \ucd phys.atmol.transition \enducd& Transition between states\\
Q & \ucd phys.atmol.transProb \enducd& Transition probability, Einstein A coefficient\\
Q & \ucd phys.atmol.wOscStrength \enducd& Weighted oscillator strength\\
Q & \ucd phys.atmol.weight \enducd& Atomic weight\\
Q & \ucd phys.columnDensity \enducd& Column density\\
S & \ucd phys.composition \enducd& Quantities related to composition of objects\\
Q & \ucd phys.composition.massLightRatio \enducd& Mass to light ratio\\
Q & \ucd phys.composition.yield \enducd& Mass yield\\
S & \ucd phys.cosmology \enducd& Related to cosmology\\
Q & \ucd phys.current \enducd& Electric current\\
Q & \ucd phys.current.density \enducd& Electric current density\\
Q & \ucd phys.damping \enducd& Generic damping quantities\\
Q & \ucd phys.density \enducd& Density (of mass, electron, ...)\\
Q & \ucd phys.density.phaseSpace \enducd& Density in the phase space\\
Q & \ucd phys.dielectric \enducd& Complex dielectric function\\
Q & \ucd phys.dispMeasure \enducd& Dispersion measure\\
S & \ucd phys.dust \enducd& Relative to dust\\
Q & \ucd phys.electCharge \enducd& Electric charge\\
V & \ucd phys.electField \enducd& Electric field\\
S & \ucd phys.electron \enducd& Electron\\
Q & \ucd phys.electron.degen \enducd& Electron degeneracy parameter\\
Q & \ucd phys.emissMeasure \enducd& Emission measure\\
Q & \ucd phys.emissivity \enducd& Emissivity\\
Q & \ucd phys.energy \enducd& Energy\\
Q & \ucd phys.energy.Gibbs \enducd& Gibbs (free) energy or free enthalpy   [ G=H –TS ]\\
Q & \ucd phys.energy.Helmholtz \enducd& Helmholtz free energy [ A=U–TS ]\\
Q & \ucd phys.energy.density \enducd& Energy density\\
Q & \ucd phys.enthalpy \enducd& Enthalpy  [ H=U+pv ]\\
Q & \ucd phys.entropy \enducd& Entropy\\
Q & \ucd phys.eos \enducd& Equation of state\\
Q & \ucd phys.excitParam \enducd& Excitation parameter U\\
E & \ucd phys.fluence \enducd& Particle energy received  by a surface per unit area and integrated over time\\
Q & \ucd phys.flux \enducd& Flux or flow of particle, energy, etc.\\
Q & \ucd phys.flux.energy \enducd& Energy flux, heat flux\\
Q & \ucd phys.gauntFactor \enducd& Gaunt factor/correction\\
Q & \ucd phys.gravity \enducd& Gravity\\
Q & \ucd phys.ionizParam \enducd& Ionization parameter\\
Q & \ucd phys.ionizParam.coll \enducd& Collisional ionization\\
Q & \ucd phys.ionizParam.rad \enducd& Radiative ionization\\
E & \ucd phys.luminosity \enducd& Luminosity\\
Q & \ucd phys.luminosity.fun \enducd& Luminosity function\\
E & \ucd phys.magAbs \enducd& Absolute magnitude\\
Q & \ucd phys.magAbs.bol \enducd& Bolometric absolute magnitude\\
V & \ucd phys.magField \enducd& Magnetic field\\
Q & \ucd phys.mass \enducd& Mass\\
Q & \ucd phys.mass.inertiaMomentum \enducd& Momentum of inertia or rotational inertia\\
Q & \ucd phys.mass.loss \enducd& Mass loss\\
Q & \ucd phys.mol \enducd& Molecular data\\
Q & \ucd phys.mol.dipole \enducd& Molecular dipole\\
Q & \ucd phys.mol.dipole.electric \enducd& Molecular electric dipole moment\\
Q & \ucd phys.mol.dipole.magnetic \enducd& Molecular magnetic dipole moment\\
Q & \ucd phys.mol.dissociation \enducd& Molecular dissociation\\
Q & \ucd phys.mol.formationHeat \enducd& Formation heat for molecules\\
Q & \ucd phys.mol.quadrupole \enducd& Molecular quadrupole\\
Q & \ucd phys.mol.quadrupole.electric \enducd& Molecular electric quadrupole moment\\
S & \ucd phys.mol.rotation \enducd& Molecular rotation\\
S & \ucd phys.mol.vibration \enducd& Molecular vibration\\
S & \ucd phys.particle \enducd& Related to physical particles\\
S & \ucd phys.particle.neutrino \enducd& Related to neutrino\\
S & \ucd phys.particle.neutron \enducd& Related to neutron\\
S & \ucd phys.particle.proton \enducd& Related to proton\\
S & \ucd phys.particle.alpha \enducd& Related to alpha particle\\
S & \ucd phys.phaseSpace \enducd& Related to phase space\\
E & \ucd phys.polarization \enducd& Polarization degree (or percentage)\\
Q & \ucd phys.polarization.circular \enducd& Circular polarization\\
Q & \ucd phys.polarization.coherency \enducd& Matrix of the correlation between components of an electromagnetic wave\\
Q & \ucd phys.polarization.linear \enducd& Linear polarization\\
Q & \ucd phys.polarization.rotMeasure \enducd& Rotation measure polarization\\
Q & \ucd phys.polarization.stokes \enducd& Stokes polarization\\
Q & \ucd phys.polarization.stokes.I \enducd& Stokes polarization coefficient I\\
Q & \ucd phys.polarization.stokes.Q \enducd& Stokes polarization coefficient Q\\
Q & \ucd phys.polarization.stokes.U \enducd& Stokes polarization coefficient U\\
Q & \ucd phys.polarization.stokes.V \enducd& Stokes polarization coefficient V\\
Q & \ucd phys.potential \enducd& Potential (electric, gravitational, etc.)\\
Q & \ucd phys.pressure \enducd& Pressure\\
Q & \ucd phys.recombination.coeff \enducd& Recombination coefficient\\
Q & \ucd phys.reflectance \enducd& Radiance factor (received radiance divided by input radiance)\\
Q & \ucd phys.reflectance.bidirectional \enducd& Bidirectional reflectance\\
Q & \ucd phys.reflectance.bidirectional.df \enducd& Bidirectional reflectance distribution function\\
Q & \ucd phys.reflectance.factor \enducd& Reflectance normalized per direction cosine of incidence angle\\
Q & \ucd phys.refractIndex \enducd& Refraction index\\
Q & \ucd phys.size \enducd& Linear size, length (not angular)\\
Q & \ucd phys.size.axisRatio \enducd& Axis ratio (a/b) or (b/a)\\
Q & \ucd phys.size.diameter \enducd& Diameter\\
Q & \ucd phys.size.radius \enducd& Radius\\
Q & \ucd phys.size.smajAxis \enducd& Linear semi major axis\\
Q & \ucd phys.size.sminAxis \enducd& Linear semi minor axis\\
Q & \ucd phys.size.smedAxis \enducd& Linear semi median axis for 3D ellipsoids\\
Q & \ucd phys.temperature \enducd& Temperature\\
Q & \ucd phys.temperature.effective \enducd& Effective temperature\\
Q & \ucd phys.temperature.electron \enducd& Electron temperature\\
Q & \ucd phys.transmission \enducd& Transmission (of filter, instrument, ...)\\
V & \ucd phys.veloc \enducd& Space velocity\\
Q & \ucd phys.veloc.ang \enducd& Angular velocity\\
Q & \ucd phys.veloc.dispersion \enducd& Velocity dispersion\\
Q & \ucd phys.veloc.escape \enducd& Escape velocity\\
Q & \ucd phys.veloc.expansion \enducd& Expansion velocity\\
Q & \ucd phys.veloc.microTurb \enducd& Microturbulence velocity\\
Q & \ucd phys.veloc.orbital \enducd& Orbital velocity\\
Q & \ucd phys.veloc.pulsat \enducd& Pulsational velocity\\
Q & \ucd phys.veloc.rotat \enducd& Rotational velocity\\
Q & \ucd phys.veloc.transverse \enducd& Transverse / tangential velocity\\
S & \ucd phys.virial \enducd& Related to virial quantities (mass, radius, ..)\\
Q & \ucd phys.volume \enducd& Volume (in cubic units)\\
Q & \ucd pos \enducd& Position and coordinates\\
Q & \ucd pos.angDistance \enducd& Angular distance, elongation\\
Q & \ucd pos.angResolution \enducd& Angular resolution\\
Q & \ucd pos.az \enducd& Position in alt-azimutal frame\\
Q & \ucd pos.az.alt \enducd& Alt-azimutal altitude\\
Q & \ucd pos.az.azi \enducd& Alt-azimutal azimut\\
Q & \ucd pos.az.zd \enducd& Alt-azimutal zenith distance\\
Q & \ucd pos.azimuth \enducd& Azimuthal angle in a generic reference plane\\
S & \ucd pos.barycenter \enducd& Barycenter\\
S & \ucd pos.bodycentric \enducd& Body-centric related coordinate\\
S & \ucd pos.bodygraphic \enducd& Body-graphic related coordinate\\
S & \ucd pos.bodyrc \enducd& Body related coordinates\\
Q & \ucd pos.bodyrc.alt \enducd& Body related coordinate (altitude on the body)\\
Q & \ucd pos.bodyrc.lat \enducd& Body related coordinate (latitude on the body)\\
Q & \ucd pos.bodyrc.lon \enducd& Body related coordinate (longitude on the body)\\
S & \ucd pos.cartesian \enducd& Cartesian (rectangular) coordinates\\
Q & \ucd pos.cartesian.x \enducd& Cartesian coordinate along the x-axis\\
Q & \ucd pos.cartesian.y \enducd& Cartesian coordinate along the y-axis\\
Q & \ucd pos.cartesian.z \enducd& Cartesian coordinate along the z-axis\\
S & \ucd pos.centroid \enducd& Related to centroid position\\
S & \ucd pos.cmb \enducd& Cosmic Microwave Background reference frame\\
S & \ucd pos.cylindrical \enducd& Related to cylindrical coordinates\\
Q & \ucd pos.cylindrical.azi \enducd& Azimuthal angle around z-axis (cylindrical coordinates)\\
Q & \ucd pos.cylindrical.r \enducd& Radial distance from z-axis (cylindrical coordinates)\\
Q & \ucd pos.cylindrical.z \enducd& Height or altitude from reference plane (cylindrical coordinates)\\
Q & \ucd pos.dirCos \enducd& Direction cosine\\
V & \ucd pos.distance \enducd& Linear distance\\
S & \ucd pos.earth \enducd& Coordinates related to Earth\\
Q & \ucd pos.earth.altitude \enducd& Altitude, height on Earth  above sea level\\
Q & \ucd pos.earth.lat \enducd& Latitude on Earth\\
Q & \ucd pos.earth.lon \enducd& Longitude on Earth\\
S & \ucd pos.ecliptic \enducd& Ecliptic coordinates\\
Q & \ucd pos.ecliptic.lat \enducd& Ecliptic latitude\\
Q & \ucd pos.ecliptic.lon \enducd& Ecliptic longitude\\
Q & \ucd pos.emergenceAng \enducd& Emergence angle of optical ray on an interface\\
S & \ucd pos.eop \enducd& Earth orientation parameters\\
Q & \ucd pos.ephem \enducd& Ephemeris\\
Q & \ucd pos.eq \enducd& Equatorial coordinates\\
Q & \ucd pos.eq.dec \enducd& Declination in equatorial coordinates\\
Q & \ucd pos.eq.ha \enducd& Hour-angle\\
Q & \ucd pos.eq.ra \enducd& Right ascension in equatorial coordinates\\
Q & \ucd pos.eq.spd \enducd& South polar distance in equatorial coordinates\\
S & \ucd pos.errorEllipse \enducd& Positional error ellipse\\
Q & \ucd pos.frame \enducd& Reference frame used for positions\\
S & \ucd pos.galactic \enducd& Galactic coordinates\\
Q & \ucd pos.galactic.lat \enducd& Latitude in galactic coordinates\\
Q & \ucd pos.galactic.lon \enducd& Longitude in galactic coordinates\\
S & \ucd pos.galactocentric \enducd& Galactocentric coordinate system\\
S & \ucd pos.geocentric \enducd& Geocentric coordinate system\\
Q & \ucd pos.healpix \enducd& Hierarchical Equal Area IsoLatitude Pixelization\\
S & \ucd pos.heliocentric \enducd& Heliocentric position coordinate (solar system bodies)\\
Q & \ucd pos.HTM \enducd& Hierarchical Triangular Mesh\\
Q & \ucd pos.incidenceAng \enducd& Incidence angle of optical ray on an interface\\
S & \ucd pos.lambert \enducd& Lambert projection\\
S & \ucd pos.lg \enducd& Local Group reference frame\\
S & \ucd pos.lsr \enducd& Local Standard of Rest reference frame\\
Q & \ucd pos.lunar \enducd& Lunar coordinates\\
Q & \ucd pos.lunar.occult \enducd& Occultation by lunar limb\\
Q & \ucd pos.nutation \enducd& Nutation (of a body)\\
Q & \ucd pos.outline \enducd& Set of points outlining a region (contour)\\
Q & \ucd pos.parallax \enducd& Parallax\\
Q & \ucd pos.parallax.dyn \enducd& Dynamical parallax\\
Q & \ucd pos.parallax.phot \enducd& Photometric parallaxes\\
Q & \ucd pos.parallax.spect \enducd& Spectroscopic parallax\\
Q & \ucd pos.parallax.trig \enducd& Trigonometric parallax\\
Q & \ucd pos.phaseAng \enducd& Phase angle, e.g. elongation of earth from sun as seen from a third cel. object\\
V & \ucd pos.pm \enducd& Proper motion\\
Q & \ucd pos.posAng \enducd& Position angle of a given vector\\
V & \ucd pos.precess \enducd& Precession (in equatorial coordinates)\\
Q & \ucd pos.resolution \enducd& Spatial linear resolution (not angular)\\
S & \ucd pos.spherical \enducd& Related to spherical coordinates\\
Q & \ucd pos.spherical.azi \enducd& Azimuthal angle (spherical coordinates)\\
Q & \ucd pos.spherical.colat \enducd& Polar or Colatitude angle (spherical coordinates)\\
Q & \ucd pos.spherical.r \enducd& Radial distance or radius (spherical coordinates)\\
S & \ucd pos.supergalactic \enducd& Supergalactic coordinates\\
Q & \ucd pos.supergalactic.lat \enducd& Latitude in supergalactic coordinates\\
Q & \ucd pos.supergalactic.lon \enducd& Longitude in supergalactic coordinates\\
P & \ucd pos.wcs \enducd& WCS keywords\\
P & \ucd pos.wcs.cdmatrix \enducd& WCS CDMATRIX\\
P & \ucd pos.wcs.crpix \enducd& WCS CRPIX\\
P & \ucd pos.wcs.crval \enducd& WCS CRVAL\\
P & \ucd pos.wcs.ctype \enducd& WCS CTYPE\\
P & \ucd pos.wcs.naxes \enducd& WCS NAXES\\
P & \ucd pos.wcs.naxis \enducd& WCS NAXIS\\
P & \ucd pos.wcs.scale \enducd& WCS scale or scale of an image\\
Q & \ucd spect \enducd& Spectroscopy\\
Q & \ucd spect.binSize \enducd& Spectral bin size\\
S & \ucd spect.continuum \enducd& Continuum spectrum\\
Q & \ucd spect.dopplerParam \enducd& Doppler parameter b\\
E & \ucd spect.dopplerVeloc \enducd& Radial velocity, derived from the shift of some spectral feature\\
E & \ucd spect.dopplerVeloc.opt \enducd& Radial velocity derived from a wavelength shift using the optical convention\\
E & \ucd spect.dopplerVeloc.radio \enducd& Radial velocity derived from a frequency shift using the radio convention\\
E & \ucd spect.index \enducd& Spectral index\\
S & \ucd spect.line \enducd& Spectral line\\
E & \ucd spect.line.asymmetry \enducd& Line asymmetry\\
E & \ucd spect.line.broad \enducd& Spectral line broadening\\
Q & \ucd spect.line.broad.Stark \enducd& Stark line broadening coefficient\\
E & \ucd spect.line.broad.Zeeman \enducd& Zeeman broadening\\
E & \ucd spect.line.eqWidth \enducd& Line equivalent width\\
E & \ucd spect.line.intensity \enducd& Line intensity\\
E & \ucd spect.line.profile \enducd& Line profile\\
Q & \ucd spect.line.strength \enducd& Spectral line strength S\\
E & \ucd spect.line.width \enducd& Spectral line full width half maximum\\
Q & \ucd spect.resolution \enducd& Spectral (or velocity) resolution\\
S & \ucd src \enducd& Observed source viewed on the sky\\
S & \ucd src.calib \enducd& Calibration source\\
S & \ucd src.calib.guideStar \enducd& Guide star\\
Q & \ucd src.class \enducd& Source classification (star, galaxy, cluster, comet, asteroid )\\
Q & \ucd src.class.color \enducd& Color classification\\
Q & \ucd src.class.distance \enducd& Distance class e.g. Abell\\
Q & \ucd src.class.luminosity \enducd& Luminosity class\\
Q & \ucd src.class.richness \enducd& Richness class e.g. Abell\\
Q & \ucd src.class.starGalaxy \enducd& Star/galaxy discriminator, stellarity index\\
Q & \ucd src.class.struct \enducd& Structure classification e.g. Bautz-Morgan\\
Q & \ucd src.density \enducd& Density of sources\\
Q & \ucd src.ellipticity \enducd& Source ellipticity\\
Q & \ucd src.impactParam \enducd& Impact parameter\\
Q & \ucd src.morph \enducd& Morphology structure\\
Q & \ucd src.morph.param \enducd& Morphological parameter\\
Q & \ucd src.morph.scLength \enducd& Scale length for a galactic component (disc or bulge)\\
Q & \ucd src.morph.type \enducd& Hubble morphological type (galaxies)\\
S & \ucd src.net \enducd& Qualifier indicating that a quantity (e.g. flux) is background subtracted rather than total\\
Q & \ucd src.orbital \enducd& Orbital parameters\\
Q & \ucd src.orbital.eccentricity \enducd& Orbit eccentricity\\
Q & \ucd src.orbital.inclination \enducd& Orbit inclination\\
Q & \ucd src.orbital.meanAnomaly \enducd& Orbit mean anomaly\\
Q & \ucd src.orbital.meanMotion \enducd& Mean motion\\
Q & \ucd src.orbital.node \enducd& Ascending node\\
Q & \ucd src.orbital.periastron \enducd& Periastron\\
Q & \ucd src.orbital.Tisserand \enducd& Tisserand parameter (generic)\\
Q & \ucd src.orbital.TissJ \enducd& Tisserand parameter with respect to Jupiter\\
Q & \ucd src.redshift \enducd& Redshift\\
Q & \ucd src.redshift.phot \enducd& Photometric redshift\\
Q & \ucd src.sample \enducd& Sample\\
Q & \ucd src.spType \enducd& Spectral type MK\\
Q & \ucd src.var \enducd& Variability of source\\
E & \ucd src.var.amplitude \enducd& Amplitude of variation\\
Q & \ucd src.var.index \enducd& Variability index\\
Q & \ucd src.var.pulse \enducd& Pulse\\
Q & \ucd stat \enducd& Statistical parameters\\
Q & \ucd stat.asymmetry \enducd& Measure of asymmetry\\
P & \ucd stat.correlation \enducd& Correlation between two parameters\\
P & \ucd stat.covariance \enducd& Covariance between two parameters\\
P & \ucd stat.error \enducd& Statistical error\\
P & \ucd stat.error.sys \enducd& Systematic error\\
Q & \ucd stat.filling \enducd& Filling factor (volume, time, ..)\\
Q & \ucd stat.fit \enducd& Fit\\
P & \ucd stat.fit.chi2 \enducd& Chi2\\
P & \ucd stat.fit.dof \enducd& Degrees of freedom\\
P & \ucd stat.fit.goodness \enducd& Goodness or significance of fit\\
S & \ucd stat.fit.omc \enducd& Observed minus computed\\
Q & \ucd stat.fit.param \enducd& Parameter of fit\\
P & \ucd stat.fit.residual \enducd& Residual fit\\
Q & \ucd stat.Fourier \enducd& Fourier coefficient\\
Q & \ucd stat.Fourier.amplitude \enducd& Amplitude of Fourier coefficient\\
S & \ucd stat.fwhm \enducd& Full width at half maximum\\
S & \ucd stat.interval \enducd& Generic interval between two limits (defined as a pair of values)\\
P & \ucd stat.likelihood \enducd& Likelihood\\
S & \ucd stat.max \enducd& Maximum or upper limit\\
S & \ucd stat.mean \enducd& Mean, average value\\
S & \ucd stat.median \enducd& Median value\\
S & \ucd stat.min \enducd& Minimum or lowest limit\\
Q & \ucd stat.param \enducd& Parameter\\
Q & \ucd stat.probability \enducd& Probability\\
P & \ucd stat.rank \enducd& Rank or order in list of sorted values\\
P & \ucd stat.rms \enducd& Root mean square as square root of sum of squared values or quadratic mean\\
P & \ucd stat.snr \enducd& Signal to noise ratio\\
P & \ucd stat.stdev \enducd& Standard deviation as the square root of the variance\\
S & \ucd stat.uncalib \enducd& Qualifier of a generic uncalibrated quantity\\
Q & \ucd stat.value \enducd& Miscellaneous value\\
P & \ucd stat.variance \enducd& Variance\\
P & \ucd stat.weight \enducd& Statistical weight\\
Q & \ucd time \enducd& Time, generic quantity in units of time or date\\
Q & \ucd time.age \enducd& Age\\
Q & \ucd time.creation \enducd& Creation time/date (of dataset, file, catalogue,...)\\
Q & \ucd time.crossing \enducd& Crossing time\\
Q & \ucd time.duration \enducd& Interval of time describing the duration of a generic event or phenomenon\\
Q & \ucd time.end \enducd& End time/date of a generic event\\
Q & \ucd time.epoch \enducd& Instant of time related to a generic event (epoch, date, Julian date, time stamp/tag,...)\\
Q & \ucd time.equinox \enducd& Equinox\\
Q & \ucd time.interval \enducd& Time interval, time-bin, time elapsed between two events, not the duration of an event\\
Q & \ucd time.lifetime \enducd& Lifetime\\
Q & \ucd time.period \enducd& Period, interval of time between the recurrence of phases in a periodic phenomenon\\
Q & \ucd time.period.revolution \enducd& Period of revolution of a body around a primary one (similar to year)\\
Q & \ucd time.period.rotation \enducd& Period of rotation of a body around its axis (similar to day)\\
Q & \ucd time.phase \enducd& Phase, position within a period\\
Q & \ucd time.processing \enducd& A time/date associated with the processing of data\\
Q & \ucd time.publiYear \enducd& Publication year\\
Q & \ucd time.relax \enducd& Relaxation time\\
Q & \ucd time.release \enducd& The time/date data is available to the public\\
Q & \ucd time.resolution \enducd& Time resolution\\
Q & \ucd time.scale \enducd& Timescale\\
Q & \ucd time.start \enducd& Start time/date of generic event\\

% /GENERATED
\sptablerule
%\end{tabular}
%\caption{List of UCD words}
\label{table:ucd-list}
\end{longtable}

\appendix

\section{Definition of atoms and words}

\subsection{Definition of atoms}
Atoms are defined following these guidelines:
\begin{enumerate}
\item Abbreviations are used in contexts where their meaning is unambiguous. ({\bf ra}, {\bf dec} are 
acceptable, but {\bf t} is ambiguous: {\bf time} and {\bf temperature} are used instead.)
\item Atoms are not hyphenated. The separation is marked by a capital letter to help readability 
(position angle = {\bf posAng}) unless the composed word has a well- known acronym (signal to noise ratio = 
{\bf snr}) or short form (standard deviation = {\bf stdev}). There are only two exceptions to this rule: 
(i) the X-ray band ({\bf em.X-ray}) and (ii) the frequency / wavelength intervals defining regions of the 
e.m.~spectrum (e.g., {\bf em.radio.3-6GHz}).
\end{enumerate}

\subsection{Definition of words}
\label{sec:words}
The list of UCD1+ words presented in this document was initially generated applying the rules and 
recommendations of PR-UCD-20040823 to catalogues/tables in VizieR. The original motivation was to 
transform old UCD1 into an improved version, trying to build a list of combinations of new words that 
could describe all the existing UCD1 terms.

The list of UCD1+ words is maintained by the UCD Scientific Board, following the procedure defined 
in the UCD Recommendation document \citep{2005ivoa.spec.0819D} and described in detail in 
\citet{2006ivoa.spec.0528M}\footnote{An earlier draft on UCD building, still at 
\url{http://www.ivoa.net/documents/PR/UCD/UCD-20040823.html} includes more details 
about the process of the change from the earlier ``UCD1'' standard, and may be of historical interest, or 
provide more rationale.}.

\section{The structure of the UCD1+ tree}
All existing UCD1+ words are grouped into 12 main categories. These categories are expressed by the 
first atom of the word, whose possible values are:
\begin{enumerate}
\item {\bf arith} (arithmetics)\\ This section includes concepts involving or indicating some 
mathematical operation performed on the primary `concept' or just the presence of an arithmetic 
factor or operator.
\item {\bf em} (electromagnetic spectrum)\\ This section describes the electromagnetic spectrum, 
either in a monochromatic way or in predefined intervals. The complete list of proposed bands (in 
seven classical regions of the electromagnetic spectrum: radio, millimetre, infrared, optical, 
ultraviolet, x-ray and gamma- ray), can be found in the document 
\url{https://wiki.ivoa.net/internal/IVOA/IvoaUCD/NoteEMSpectrum-20040520.html}
\item {\bf instr} (instrument)\\ This section includes all quantities related to astronomical 
instrumentation, e.g. detectors (plates, CCDs, etc.), spectrographs, and telescopes (including 
observatories or missions), etc.
\item {\bf meta} (metadata)\\ This section includes all the information that is not coming directly 
from a measurement, and information that could not be included in other sections.
\item {\bf obs} (observation)\\ In principle under this section should go all words describing an 
observation (the name of the observer or PI, the observing conditions, the name of the field). In 
practice, this section helps to identify concepts related to an observation process.
\item {\bf phot} (photometry)\\ All the words describing photometric measures are included in this 
section. The definitions distinguish between a flux density (flux per unit frequency interval), a 
flux density integrated over a given electromagnetic spectrum interval (flux if expressed linearly, 
mag if expressed by a log), or a flux expressed in counts/s (if the setup of the detector is photon 
counting observing mode). `Colors', which are differences of magnitudes (i.e. ratios of fluxes) 
measured in different bandpasses, are also included.
\item {\bf phys} (physics)\\ This section includes atomic and molecular data (mainly used for 
spectroscopy) and basic physical quantities (temperature, mass, gravity, luminosity, etc.)
\item {\bf pos} (positional data)\\ This section describes all quantities related to the position of 
an object on the sky:
\begin{itemize}
\item Angular coordinates, and projections from spherical to rectangular systems.
\item Angular measurements in general (the angular size of an object is in this section, its linear 
size is in the {\bf phys} section).
\item The World Coordinate System FITS keywords.
\end{itemize}
\item {\bf spect} (spectral data)\\ For historical reasons, photometric data taken in narrow spectral 
bands with instruments called spectrographs are classified as spectroscopic data. These definitions 
should not be confused with those in the {\bf em} category. {\bf em} represents the independent 
variable, or dispersion axis, and {\bf phot} and {\bf spect} describe the dependent measures like a 
flux under the {\bf phot} branch, and spectral measures spectral line physical features one can 
measure on a spectrum, for instance, under the {\bf spect} branch. 
\item {\bf src} (source)\\ This is a rather generic section, mainly devoted to source classifications.
Variability, orbital, and velocity data are also included in this section.
\item {\bf stat} (statistics)\\ This section includes statistical information on measurements.
\item {\bf time} (time)\\ Quantities related to time (age, date, period, etc.) are described in this 
section.
\end{enumerate}

\section{Combining UCD words}
\subsection{Goal}
Since their definition UCDs have been used in major catalogue archives, in the definition of various
VO protocols (SSA, SIAv2, SLAP, TAP ObsTAP, EPN-TAP, etc.) and used with success to provide semantic
annotation for a huge collection of table columns distributed in the astronomical community.

The list of terms has increased and the usage of UCD combination has become very common. This leads
to a richer set of rules in the assigning and checking tools developed at CDS with VO partners.

In order to keep the consistency in the UCD thesaurus, each rule is adjusted and weighted considering 
the physical usage of the quantities represented in table columns, so the pertinence increases with 
the context. Initially used for source catalogues (Vizier, Heasarc archives, etc.) in the first place, 
they are now also used in VOTable documents for planetary data \citep{wd:epntap,erard-vespa} and all sorts of metadata.

\subsection{Remarks on combination rules}
The combination rules have been defined in the first IVOA documents defining UCD concept
\citep{2005ivoa.spec.0819D} and refined in the last UCD1+ standard recommendation \citep{2018ivoa.spec.0527M}. They are exposed with a syntax tag given as a property of each UCD word 
and included in the list of UCD words. See Section \ref{sec:list} with the tags definitions on top.

They correspond to real usage of the terms in science publications and are attached to the description 
of catalogues’ column by experimented data scientists. UCD combination also reflects the catalogues 
build-up strategy. Errors and statistics, for instance, are provided with measurement values; measures 
and model comparison are evaluated with error fits, precision, etc. All the scientific knowledge helps 
to define appropriate UCD words combination.

The assigning tool proposed at \url{http://cds.u-strasbg.fr/UCD/cgi-bin/descr2ucd} is based on the
pragmatic encoding of physical quantities found in science papers and data attached to publications.

\section{Current questions about combinations of UCDs}
\subsection{How do UCDs differ from structured descriptions?}

UCDs do not provide a structured representation of table content but the meaning or relative class 
concept known at the time for the astronomical speciality. Therefore, the structure of words and 
their rules for combination do not follow any object oriented paradigm, in contradiction to any 
reference to a data model item (Utype, VO-DML type /role definition), which are dependent of a 
defined and endorsed IVOA data model specification.

\subsection{P or S syntax code: Which is the most pertinent position for a UCD word?}
P, S and Q are the labels expressing in which position of a UCD expression a term can be used, P 
in first place, S as suffix, and Q in both allowed position: head and tail. The UCD list defines 
the recommended position for each word with some flexibility.

P is always what matters the most to describe a quantity, i.e., the kind of property that should be 
searched for in primary order, and the most relevant UCD words to represent a quantity.

S is the code for the qualifying part of the UCD, the secondary information appended to specify 
the first UCD term.

%All the codes are explained and given in the list of UCD terms in Appendix A. 
Examples:
\begin{itemize}
\item Give me all columns / all catalogues with a column having a magnitude in R: {\tt magnitude} 
is the primary concept and band R is the secondary concept, so the ucd to search for is 
{\tt phot.mag;em.opt.R}.
\item Give me all columns with an error on magnitude B: here we shall use a query with ucd 
equals to {\tt stat.error;phot.mag;em.opt.B}. Here the main concept attached to the column 
value is error, qualified by {\tt phot.mag}, itself qualified by {\tt em.opt.B}.
\end{itemize}
Concatenation can apply more than one time, depending on ordering rules. See \citet{2005ivoa.spec.0819D}, section 3.3, 
for other details.


\section{Changes from previous versions}

\subsection{Changes from REC v1.3 following RFM}
The document title has been updated and the document sections have been reorganised to focus on 
the list of terms, moving UCD1+ standard reminders to Appendix sections. 

The modifications decided during the UCD1+ list v1.3 
RFM\footnote{\url{https://wiki.ivoa.net/twiki/bin/view/IVOA/UCDList_1-3_RFM}} process are presented below.

%\subsubsection*{Deletion}
%\subsubsection*{Ammendment}
\subsubsection*{Additions}
\footnotesize\begin{longtable}[h!]{c|p{39.5ex}|D{0.45\textwidth}}
\sptablerule
Q & {\tt phys.electCharge} & Electric charge\\
Q & {\tt phys.current} & Electric current\\
Q & {\tt phys.current.density} & Electric current density\\
Q & {\tt pos.incidenceAng} & Incidence angle of optical ray on an interface\\
Q & {\tt pos.emergenceAng} & Emergence angle of optical ray on an interface\\
Q & {\tt pos.azimuth} & azimuthal angle in a generic reference plane\\
Q & {\tt phys.reflectance} & Radiance factor (received radiance divided by input radiance)\\
Q & {\tt phys.reflectance.bidirectional} & Bidirectional reflectance\\
Q & {\tt phys.reflectance.bidirectional.df} & Bidirectional reflectance distribution function\\
Q & {\tt phys.reflectance.factor} & Reflectance normalized per direction cosine of incidence angle\\
S & {\tt pos.cylindrical} & Related to cylindrical coordinates\\
Q & {\tt pos.cylindrical.r} & Radial distance from z-axis (cylindrical coordinates)\\
Q & {\tt pos.cylindrical.azi} & Azimuthal angle around z-axis (cylindrical coordinates)\\
Q & {\tt pos.cylindrical.z} & Height or altitude from reference plane (cylindrical coordinates)\\
S & {\tt pos.spherical} & Related to spherical coordinates\\
Q & {\tt pos.spherical.r} & Radial distance or radius (spherical coordinates)\\
Q & {\tt pos.spherical.colat} & Polar or Colatitude angle (spherical coordinates)\\
Q & {\tt pos.spherical.azi} & Azimuthal angle (spherical coordinates)\\
Q & {\tt pos.resolution} & Spatial linear resolution (not angular)\\
S & {\tt pos.bodycentric} & Body-centric related coordinate\\
S & {\tt pos.bodygraphic} & Body-graphic related coordinate\\
Q & {\tt meta.checksum} & Numerical signature of digital data\\
Q & {\tt phys.polarization.coherency} & Matrix of the correlation between components of an electromagnetic wave\\
\sptablerule
\end{longtable}
\subsubsection*{Clarification}
%text improvement 
Clarified position rules for syntax code E, C, V in Appendix B.  

\subsection{Changes from PR v1.3-2018 following TCG comments}
\subsubsection*{Update of definitions} 
\footnotesize\begin{longtable}[h!]{c|p{39.5ex}|D{0.45\textwidth}}
\sptablerule
Q & {\tt meta.query} &  A query posed to an information system or database or a property of it\\
\sptablerule
\end{longtable}

\subsubsection*{Changes of position indicator} 
Was
\footnotesize\begin{longtable}[h!]{c|p{39.5ex}|D{0.45\textwidth}}
\sptablerule
Q & {\tt phys.atmol.collisional} & Related to collisions\\
Q & {\tt phys.virial}            & Related to virial quantities (mass, radius,..)\\
\sptablerule
\end{longtable}
Changed to S to conform to the “Related to” definition and the usage of this UCD, mostly appearing as suffix.  
\footnotesize\begin{longtable}[h!]{c|p{39.5ex}|D{0.45\textwidth}}
\sptablerule
S & {\tt phys.atmol.collisional} & Related to collisions\\
S & {\tt phys.virial}            & Related to virial quantities (mass, radius,..)\\
\sptablerule
\end{longtable}

\subsection{Changes from WD v1.3-20160719}
Added section 3 Remarks on combination rules for UCD words.

\subsubsection*{New terms}
\footnotesize\begin{longtable}[h!]{c|p{39.5ex}|D{0.45\textwidth}}
\sptablerule
P & {\tt meta.ref.doi} & DOI identifier (dereferenceable)\\
\sptablerule
\end{longtable}

\subsection{Changes from WD v1.23-20160719}
\subsubsection*{Additions}
\footnotesize\begin{longtable}[h!]{c|p{39.5ex}|D{0.45\textwidth}}
\sptablerule
S & {\tt arith.squared} & Squared quantity\\
S & {\tt arith.sum} & Summed or integrated quantity\\
S & {\tt arith.variation} & Generic variation of a quantity\\
S & {\tt instr.voxel} & Related to a voxel ( n-D volume element with n>2)\\
Q & {\tt pos.outline} & Set of points outlining a region (contour)\\
Q & {\tt stat.asymmetry} & Measure of asymmetry\\
Q & {\tt phys.polarization.stokes.I} & Stokes polarization coefficient I\\
Q & {\tt phys.polarization.stokes.Q} & Stokes polarization coefficient Q\\
Q & {\tt phys.polarization.stokes.U} & Stokes polarization coefficient U\\
Q & {\tt phys.polarization.stokes.V} & Stokes polarization coefficient V\\
Q & {\tt stat.asymmetry} & Measure of asymmetry\\
S & {\tt stat.fwhm} & Full width at half maximum\\
S & {\tt stat.interval} & Generic interval between two limits (defined as a pair of values)\\
P & {\tt stat.rank} & Rank or order in list of sorted value\\
P & {\tt stat.rms} & Root mean square  Square root of sum of squared values or quadratic mean\\
\sptablerule
\end{longtable}

\subsubsection*{Amendments/clarifications}
Definition for 
\begin{itemize}
\item {\tt phys.area} Area (in surface, not angular units)
\item {\tt stat.stdev} Standard deviation as the square root of the variance                                                           
\end{itemize}

\subsection{Changes from WD v1.23-20150608}
Text of Abstract, last two lines. Added reference to \citet{tn:solar-system-ucd}.

Section 1.1 Definition: ``Abbreviations are used in contexts where their meaning is unambiguous'' instead of ``kept to a minimum...''

\subsubsection*{Amendments/clarifications}
\begin{flushleft}
Description changed in words: 
{\tt em.UV.10-50nm}, {\tt em.UV.100-200nm}, {\tt em.UV.200-300nm}, {\tt meta.id.PI}, 
{\tt phot.flux}, {\tt phot.fluence}, {\tt src.class}.
\end{flushleft}

\subsubsection*{Additions}
\footnotesize\begin{longtable}[h!]{c|p{39.5ex}|D{0.45\textwidth}}
\sptablerule
Q & {\tt em.freq.cutoff} & Cutoff frequency\\
Q & {\tt em.freq.resonance} & Resonance frequency\\
S & {\tt em.pw} & Plasma waves (trapped in local medium)\\
S & {\tt em.radio.20MHz} & Radio below 20 MHz\\
Q & {\tt instr.experiment} & Experiment or group of instruments\\
Q & {\tt meta.calibLevel} & Processing/calibration level\\
S & {\tt meta.preview} & Related to a preview operation (for a dataset)\\
Q & {\tt meta.query} & Related to query posed to an information system or database\\
Q & {\tt meta.ref.ivoid} & An identifier following the IVOA Identifiers recommendation\\
S & {\tt obs.calib.dark} & Related to dark current calibration\\
S & {\tt obs.occult} & Observation of occultation phenomenon by solar system objects\\
S & {\tt obs.transit} & Observation of transit phenomenon: exo-planets\\
E & {\tt phot.radiance} & Radiance as energy flux per solid angle\\
S & {\tt phys.aerosol} & Relative to aerosol\\
Q & {\tt phys.density.phaseSpace} & Density in the phase space\\
S & {\tt phys.dust} & Relative to dust\\
E & {\tt phys.fluence} & Radiant photon energy received by a surface per unit area, or irradiance of a surface integrated over time of irradiation\\
Q & {\tt phys.flux} & Flux or flow of particle, energy, etc.\\
Q & {\tt phys.flux.energy} & Energy flux, heat flux\\
Q & {\tt phys.mass.inertiaMomentum} & Momentum of inertia or rotational inertia\\
S & {\tt phys.particle} & Related to physical particles\\
S & {\tt phys.particle.neutron} & Related to neutron\\
S & {\tt phys.particle.proton} & Related to proton\\
S & {\tt phys.particle.alpha} & Related to alpha particle\\
S & {\tt phys.phaseSpace} & Related to phase space\\
Q & {\tt phys.potential} & Potential (electric, gravitational, etc.)\\
Q & {\tt phys.size.smedAxis} & Linear semi median axis for 3D ellipsoids\\
Q & {\tt phys.volume} & Volume (in cubic units)\\
Q & {\tt pos.outline} & Set of points outlining a region (contour)\\
Q & {\tt src.orbital.Tisserand} & Tisserand parameter (generic)\\
Q & {\tt src.orbital.TissJ} & Tisserand parameter with respect to Jupiter\\
Q & {\tt time.period.revolution} & Period of revolution of a body around a primary one (similar to year)\\
Q & {\tt time.period.rotation} & Period of rotation of a body around its axis (similar to day)\\
\sptablerule
\end{longtable}

\subsubsection*{Deletions/replacements}
\begin{itemize}
\item deleted: {\tt em.UV.FUV} 
\item deleted: {\tt phys.mol.qn}; replaced by: {\tt phys.atmol.qn}
\item deleted: {\tt pos.bodyrc.long}; replaced by: {\tt pos.bodyrc.lon}
\item deleted: {\tt pos.eop.nutation}; replaced by: {\tt pos.nutation}
\end{itemize}

\subsubsection*{Deprecated}
\begin{itemize}
\item {\tt meta.ref.ivorn}: The term IVORN should not be used any more for IVOA Identifiers (IVOIDs). 
In UCDs, {\tt meta.ref.ivoid} should be used instead.
\end{itemize} 

\subsection{Changes from PR v1.22}

Text of pararagraph 1.1 (2), last three lines;

List of {\tt em} bands reordered according to wavelength/frequencies.

\subsubsection*{Amendments/clarifications}
Description changed in words: {\tt phys.atmol.qn}

\subsubsection*{Additions}
{\tt em.line.Hdelta}, {\tt em.line.Lyalpha}, {\tt em.line.CO}.

\subsubsection*{Deletions/replacements}
\begin{itemize}
\item deleted: {\tt phys.mol.qn}; replaced by: {\tt phys.atmol.qn}
\end{itemize}

\subsection{Changes from PR v1.21}
\subsubsection*{Amendments/clarifications}
\begin{itemize}
\item Syntax flag changed in words: {\tt phys.polarization}
\item \begin{flushleft}
Description changed in words: 
{\tt em.IR.FIR}, {\tt em.IR.MIR}, {\tt em.IR.NIR}, {\tt em.line.OIII}
\end{flushleft}
\end{itemize}

\subsection{Changes from PR v1.2}
\subsubsection*{Additions}
\begin{itemize}
\item {\tt spect.continuum}
\end{itemize}

\subsection{Changes from REC v1.11 (Rec20051231)}
\subsubsection*{Amendments/clarifications}
\begin{itemize}
\item Spelling: {\tt phys.atmol.sWeight}
\item Syntax flag changed in words: {\tt phys.atmol}, {\tt spect.line}
\item \begin{flushleft}
Description changed in words: 
{\tt meta.dataset}, {\tt obs.atmos}, {\tt phot.color.reddFree}, 
{\tt phys.size}, {\tt phys.size.diameter}, {\tt phys.size.radius}, {\tt stat.param}, {\tt stat.value}, 
{\tt time}, {\tt time.epoch}, {\tt time.interval}, {\tt time.period}, {\tt time.phase}.
\end{flushleft}
\end{itemize}

\subsubsection*{Additions}
\begin{flushleft}
{\tt em.bin}, {\tt em.binSize}, {\tt em.IR.FIR}, {\tt em.IR.MIR}, {\tt em.IR.NIR}, {\tt em.UV.FUV}, 
{\tt meta.abstract}, {\tt meta.code.status}, {\tt meta.email}, {\tt meta.id.PI}, {\tt meta.id.CoI}, 
{\tt meta.ref.ivorn}, {\tt meta.ref.uri}, {\tt obs.calib.flat}, {\tt obs.exposure}, {\tt obs.proposal}, 
{\tt obs.proposal.cycle}, {\tt obs.sequence}, {\tt phys.atmol.symmetry}, {\tt phys.atmol.sWeight.nuclear}, 
{\tt phys.cosmology}, {\tt phys.damping}, {\tt phys.entropy}, {\tt phys.particle.neutrino}, {\tt phys.virial}, 
{\tt spect.line.strength}, {\tt src.calib}, {\tt src.calib.guideStar}, {\tt src.net}, {\tt stat.filling}, 
{\tt stat.probability}, {\tt stat.uncalib}, {\tt time.creation}, {\tt time.duration}, {\tt time.end}, 
{\tt time.processing}, {\tt time.publiYear}, {\tt time.release}, {\tt time.star}
\end{flushleft}

\subsubsection*{Deletions/replacements}
\begin{itemize}
\item deleted: {\tt phys.atmol.damping}; replaced by: {\tt phys.damping} with description: Atomic damping quantities (van der Waals)
\item deleted: {\tt phys.atmol.qn.I}; replaced by: {\tt phys.atmol.qn} with description: Nuclear spin quantum number
\item deleted: {\tt time.event}; replaced by: {\tt time.duration} with description: Duration of an event or phenomenon
\item deleted: {\tt time.event.end}; replaced by: {\tt time.end} with description: End time of event or phenomenon
\item deleted: {\tt time.event.start}; replaced by: {\tt time.start} with description: Start time of event or phenomenon
\item deleted: {\tt time.expo}; replaced by: {\tt time.duration;obs.exposure} with description: Exposure on-time, duration
\item deleted: {\tt time.expo.end}; replaced by: {\tt time.end;obs.exposure} with description: End time of exposure
\item deleted: {\tt time.expo.start}; replaced by: {\tt time.start;obs.exposure} with description: Start time of exposure
\item deleted: {\tt time.obs}; replaced by: {\tt time.duration;obs} with description: Observation on-time, duration
\item deleted: {\tt time.obs.end}; replaced by: {\tt time.end;obs} with description: End time of observation
\item deleted: {\tt time.obs.start}; replaced by: {\tt time.start;obs} with description: Start time of observation
\end{itemize}

\subsection{Changes from v1.10}
\begin{enumerate}
\item A few minor changes to the text have been done
\item All UCD words are now compliant with the UCD recommendation. The corresponding changes are described below
\item The following words have been deprecated:

\begin{tabular}{|l|l|}
\sptablerule
Deprecated UCD & New corresponding UCD\\
\sptablerule
{\tt phot.fluxDens} & {\tt phot.flux.density}\\
{\tt phot.fluxDens.sb} & {\tt phot.flux.density.sb}\\ 
{\tt phys.at*} & {\tt phys.atmol*}\\
{\tt phys.atmol.coll} & {\tt phys.atmol.collisional}\\
{\tt phys.atmol.ion} & {\tt phys.atmol.ionStage}\\
{\tt phys.atmol.trans} & {\tt phys.atmol.transition}\\
{\tt phys.energyDensity} & {\tt phys.energy.density}\\
{\tt phys.massToLight} & {\tt phys.composition.massLightRatio}\\
{\tt phys.massYield} & {\tt phys.composition.yield}\\
{\tt spect.doppler} & {\tt spect.dopplerParam}\\
\sptablerule
\end{tabular}

\item The following word has been created: {\tt phys.composition}
\item The section Changes from previous versions has been reformatted
\end{enumerate}

\subsection{Changes from v1.0}
\begin{enumerate}
\item Descriptions have been changed for the following words: {\tt em.line}, {\tt instr.pixel}, 
{\tt phys.gravity}, {\tt pos.earth.altitude}
\item The syntax flags changed for words: {\tt instr.filter}, {\tt phys.angSize}
\item The following words have been deprecated:

\begin{tabular}{|l|l|}
\sptablerule
Deprecated UCD & New corresponding UCD\\
\sptablerule
{\tt instr.filter.transm} & {\tt phys.transm;instr.filter}\\
{\tt phys.mass.light} & {\tt phys.massToLight}\\
{\tt pos.resolution} & {\tt pos.angResolution}\\
{\tt pos.satellite} & {\tt pos.bodyrc}\\
\sptablerule
\end{tabular}

\item \begin{flushleft}
The following words have been created: 
{\tt phys.polarization.circular}, 
{\tt phys.polarization.linear}, {\tt phys.size.axisRatio}, {\tt pos.bodyrc.alt}, 
{\tt pos.bodyrc.lat}, {\tt pos.bodyrc.long}, {\tt time.event}, {\tt time.event.end}, 
{\tt time.event.start}.
\end{flushleft}
\end{enumerate}

\subsection{Changes from v1.01}

The following words have been restored to their previous spelling (v1.00): 
\begin{flushleft}
{\tt phot.fluDensity}, 
{\tt phys.energDensity}, {\tt phys.mYield}, {\tt phot.fluxDensity}, {\tt phys.energyDensity}, 
{\tt phys.massYield}.
\end{flushleft}

A note has been added to indicate that these words do not strictly comply with the UCD1+ Rec.

\subsection{Changes from v1.00}
\begin{enumerate}
\item \begin{flushleft}
Descriptions have been changed for the following words: 
{\tt em.IR.H}, {\tt em.IR.J}, 
{\tt em.IR.K}, {\tt em.X-ray.hard}, {\tt em.X-ray.medium}, {\tt em.X-ray.soft}, {\tt em.gamma.hard}, 
{\tt em.gamma.soft}, {\tt em.opt.B}, {\tt em.opt.I}, {\tt em.opt.R}, {\tt em.opt.U}, {\tt em.opt.V}, 
{\tt instr.bandpass}, {\tt phot.count}, {\tt phys.density}, {\tt phys.mol.dipole.electric}, 
{\tt phys.mol.dipole.magnetic}, {\tt phys.mol.quadrupole.electric}, {\tt pos.angDistance}, 
{\tt pos.precess}, {\tt src}, {\tt src.class.distance}, {\tt src.class.richness}, 
{\tt src.class.starGalaxy}, {\tt src.class.struct}, {\tt time.expo}, {\tt time.expo.end}, 
{\tt time.expo.start}, {\tt time.interval}
\end{flushleft}
\item The following words have been deprecated: 

\begin{longtable}{|l|l|}
\sptablerule
Deprecated UCD & New corresponding UCD\\
\sptablerule
{\tt instr.angRes} & {\tt pos.resolution}\\
{\tt instr.obsty.site} & {\tt pos.earth.altitude;instr.obsty}\\ 
{\tt instr.obsty.site.seeing} & {\tt instr.obsty.seeing}\\
{\tt instr.spect} & {\tt instr}\\
{\tt instr.spect.dispersion} & {\tt instr.dispersion}\\
{\tt instr.spect.order} & {\tt instr.order}\\
{\tt instr.spect.resolution} & {\tt spect.resolution}\\
{\tt instr.tel.focus} & {\tt instr.tel.focalLength}\\
{\tt meta.fits.software} & {\tt meta.software }\\
{\tt obs.air} & {\tt obs.atmos}\\
{\tt obs.air.extinction} & {\tt obs.atmos.extinction}\\
{\tt obs.air.mass} & {\tt obs.airMass}\\
{\tt phot.fluxDens} & {\tt phot.fluDens}\\
{\tt phot.fluxDens.sb} & {\tt phot.fluDens.sb}\\ 
{\tt phot.sb} & {\tt phot.mag.sb}\\
{\tt phys.at.branchingRatio} & {\tt phys.atmol.branchingRatio}\\
{\tt phys.at.crossSection} & {\tt phys.atmol.crossSection}\\
{\tt phys.at.lineShift} & {\tt phys.atmol.lineShift}\\
{\tt phys.at.moment} & \\
{\tt phys.at.moment.electric} & {\tt phys.at.radiationType}\\ 
{\tt phys.at.moment.magnetic} & {\tt phys.at.radiationType}\\
{\tt phys.at.qn.S} & {\tt phys.at.qn}\\
{\tt phys.at.qn.L} & {\tt phys.at.qn}\\
{\tt phys.at.qn.J} & {\tt phys.at.qn}\\
{\tt phys.at.qn.F} & {\tt phys.at.qn}\\
{\tt phys.atmol.state.final} & {\tt  phys.atmol.final}\\
{\tt phys.atmol.state.initial} & {\tt phys.atmol.initial}\\
{\tt phys.massYield} & {\tt phys.mYield}\\
{\tt phys.mol.quadrupole.magnetic} & {\tt phys.at.radiationType}\\
{\tt phys.refraction} & {\tt phys.refractIndex}\\
{\tt pos.az.ha} & {\tt pos.eq.ha}\\
{\tt pos.earth.nutation} & {\tt pos.eop.nutation}\\
{\tt spect.veloc} & {\tt spect.dopplerVeloc}\\
{\tt src.fwhm} & {\tt phys.angSize;src}\\
{\tt src.orbital.veloc} & {\tt phys.veloc.orbital}\\
{\tt src.veloc} & {\tt phys.veloc}\\
{\tt src.veloc.ang} & {\tt phys.veloc.ang}\\
{\tt src.veloc.cmb} & {\tt phys.veloc;pos.cmb}\\ 
{\tt src.veloc.dispersion} & {\tt phys.veloc.dispersion}\\
{\tt src.veloc.escape} & {\tt phys.veloc.escape}\\
{\tt src.veloc.expansion} & {\tt phys.veloc.expansion}\\ 
{\tt src.veloc.lg} & {\tt phys.veloc;pos.lg}\\
{\tt src.veloc.lsr} & {\tt phys.veloc;pos.lsrv}\\
{\tt src.veloc.microTurb} & {\tt phys.veloc.microTurb}\\
{\tt src.veloc.pulsat} & {\tt phys.veloc.pulsat}\\
{\tt src.veloc.rotat} & {\tt phys.veloc.rotat}\\
\sptablerule
\end{longtable}

\item \begin{flushleft}
The syntax flags changed for words: 
{\tt instr.fov}, {\tt instr.obsty}, {\tt meta.file}, 
{\tt phys.angSize}, {\tt pos.cartesian}, {\tt stat.fit.omc}
\end{flushleft}
\item \begin{flushleft}
The following words have been created: 
{\tt instr.dispersion}, {\tt instr.order}, {\tt 
instr.tel.focalLength}, {\tt meta.curation}, {\tt meta.software}, {\tt meta.version}, {\tt 
obs.atmos}, {\tt obs.atmos.extinction}, {\tt obs.airMass}, {\tt obs.atmos.refractAngle}, 
{\tt obs.calib}, {\tt phys.at.radiationType}, {\tt phys.atmol.branchingRatio}, {\tt 
phys.atmol.crossSection}, {\tt phys.atmol.lifetime}, {\tt phys.atmol.lineShift}, {\tt 
phys.energDensity}, {\tt phys.refractIndex}, {\tt phys.transmission}, {\tt pos.eq.ha}, 
{\tt pos.az.azi}, {\tt pos.bodyrc}, {\tt pos.cmb}, {\tt pos.earth.altitude}, {\tt pos.eop}, 
{\tt pos.eop.nutation}, {\tt pos.lg}, {\tt pos.lsr}, {\tt pos.phaseAng}, {\tt pos.resolution}, 
{\tt spect.resolution}, {\tt spect.dopplerVeloc}, {\tt spect.dopplerVeloc.radio}, {\tt 
spect.dopplerVeloc.opt}, {\tt src.orbital.meanMotion}, {\tt phys.veloc}, {\tt phys.veloc.ang}, 
{\tt phys.veloc.dispersion}, {\tt phys.veloc.escape}, {\tt phys.veloc.expansion}, {\tt 
phys.veloc.microTurb}, {\tt phys.veloc.orbital}, {\tt phys.veloc.pulsat}, {\tt phys.veloc.rotat}, 
{\tt phys.veloc.transverse}, {\tt time.obs}, {\tt time.obs.end}, {\tt time.obs.start}.
\end{flushleft}
\end{enumerate}

\subsection{Changes from v0.2}
\begin{enumerate}
\item Section 1.2 has been simplified
\item new syntax codes (E, C, V) have been introduced, and described in appendix A
\item The following words have been renamed

\begin{longtable}{|l|l|}
\sptablerule
Deprecated UCD & New corresponding UCD\\
\sptablerule
{\tt em.line.21cm} & {\tt em.line.HI}\\
{\tt instr.ang-res} & {\tt instr.angRes}\\
{\tt instr.sky-level} & {\tt instr.skyLevel}\\
{\tt instr.sky-temp} & {\tt instr.skyTemp}\\
{\tt instr.antenna-temp} & {\tt phot.antennaTemp}\\
{\tt phys.absorption.gf} & {\tt phys.gauntFactor}\\
{\tt phys.at.einstein} & {\tt phys.at.transProb}\\
{\tt phys.at.level} & {\tt phys.atmol.level}\\
{\tt phys.dispMeas} & {\tt phys.dispMeasure}\\ 
{\tt phys.distance} & {\tt pos.distance}\\
{\tt phys.polarization.rotMeas} & {\tt phys.polarization.rotMeasure}\\
{\tt phys.size.area} & {\tt phys.area}\\
{\tt pos.ang.separation} & {\tt pos.angDistance}\\ 
{\tt pos.ec} & {\tt pos.ecliptic}\\
{\tt pos.ec.lat} & {\tt pos.ecliptic.lat}\\
{\tt pos.ec.lon} & {\tt pos.ecliptic.lon}\\
{\tt pos.ee} & {\tt pos.errorEllipse}\\ 
{\tt pos.gal} & {\tt pos.galactic}\\
{\tt pos.gal.lat} & {\tt pos.galactic.lat}\\
{\tt pos.gal.lon} & {\tt pos.galactic.lon}\\
{\tt pos.sg} & {\tt pos.supergalactic}\\
{\tt pos.sg.lat} & {\tt pos.supergalactic.lat}\\
{\tt pos.sg.lon} & {\tt pos.supergalactic.lon}\\
{\tt src.class.star-galaxy} & {\tt src.class.starGalaxy}\\
\sptablerule
\end{longtable}

\item \begin{flushleft}
The following words have been created: 
{\tt instr.beam}, {\tt meta.code.error}, {\tt meta.id.part}, {\tt phot.flux.sb}, 
{\tt phys.angArea}, {\tt phys.angSize}, {\tt phys.angSize.smajAxis}, {\tt phys.angSize.sminAxis}, {\tt phys.area}, 
{\tt phys.at.damping}, {\tt phys.at.weight}, {\tt phys.atmol.excitation}, {\tt phys.mol.dissociation}, {\tt 
phys.recombination.coeff}, {\tt phys.size.smajAxis}, {\tt phys.size.sminAxis}, {\tt pos.cartesian}, {\tt pos.cartesian.x},
{\tt pos.cartesian.y}, {\tt pos.cartesian.z}, {\tt pos.distance}, {\tt pos.eq.spd}, {\tt pos.galactocentric}, {\tt 
pos.geocentric}, {\tt pos.healpix}, {\tt pos.heliocentric}, {\tt pos.HTM}, {\tt pos.lambert}, {\tt pos.satellite}, 
{\tt spect.line.broad.Stark}, {\tt spect.veloc}, {\tt src.redshift.phot}, {\tt stat.correlation}, {\tt time.lifetime}.
\end{flushleft}
\item Some words have been removed. The following table summarizes, when relevant, the suggested replacement to be used. 

\begin{longtable}{|l|l|}
\sptablerule
Deprecated UCD & New corresponding UCD\\
\sptablerule
{\tt instr.area} & {\tt phys.area;instr}\\
{\tt instr.beam-width} & {\tt phys.angSize;instr.beam}\\
{\tt meta.table.axis} & {\tt phys.size;meta.table}\\
{\tt phot.color.Cous} & {\tt phot.color}\\
{\tt phot.color.Gen} & {\tt phot.color}\\
{\tt phot.color.Gunn} & {\tt phot.color}\\
{\tt phot.color.JHN} & {\tt phot.color}\\
{\tt phot.color.STR} & {\tt phot.color}\\
{\tt phot.color.STR.c1} & {\tt phot.color}\\ 
{\tt phot.color.STR.b-y} & {\tt phot.color}\\
{\tt phot.color.STR.m1} & {\tt phot.color}\\
{\tt phys.at.lineBroad} & {\tt spect.line.broad}\\ 
{\tt phys.distance.compon} & {\tt pos.distance;pos.cartesian.x} (or {\tt y}, {\tt z})\\
{\tt phys.distance.gc} & {\tt pos.distance;pos.galactocentric}\\
{\tt phys.electron.energy} & {\tt phys.energy;phys.electron}\\
{\tt phys.extension} & {\tt phys.angSize or phys.size}\\
{\tt phys.mass.fraction} & {\tt phys.mass;arith.ratio}\\
{\tt phys.polarization.posAng} & {\tt pos.posAng;phys.polarization}\\
{\tt pos.ang} & \\
{\tt pos.det} & {\tt pos.cartesian;instr.det}\\
{\tt pos.eq.dec.arcsec} & \\
{\tt pos.eq.ra.minutes} & \\
{\tt pos.eq.ra.seconds} & \\
{\tt pos.gal.compon} & {\tt pos.cartesian;pos.galactic}\\
{\tt pos.pm.dec} & {\tt pos.pm;pos.eq.dec}\\
{\tt pos.pm.ra} & {\tt pos.pm;pos.eq.ra}\\
{\tt pos.precess.dec} & {\tt pos.precess;pos.eq.dec}\\
{\tt pos.precess.ra} & {\tt pos.precess;pos.eq.ra}\\
{\tt pos.proj} & \\
{\tt pos.sg.compon} & {\tt pos.cartesian;pos.supergalactic}\\
{\tt src.orbital.energy} & {\tt phys.energy;src.orbital}\\
{\tt src.orbital.separation} & {\tt pos.angDistance;src.orbital}\\
{\tt src.orbital.size} & {\tt phys.size;src.orbital}\\
{\tt src.separation} & {\tt pos.angDistance;src}\\
{\tt src.veloc.compon} & {\tt src.veloc;pos.cartesian}\\ 
{\tt src.veloc.gc} & {\tt src.veloc;pos.galactocentric}\\
{\tt src.veloc.geoc} & {\tt src.veloc;pos.geocentric}\\
{\tt src.veloc.hc} & {\tt src.veloc;pos.heliocentric}\\
\sptablerule
\end{longtable}
\end{enumerate}

\subsection{Changes from v0.1}
\begin{enumerate}
\item Descriptions of the words were improved.
\item Designation of commonly used lines have been moved to {\tt em.line.*}. As a consequence, terms like 
{\tt em.IR.K.Brgamma} or {\tt spect.index.Hbeta} have been removed.
\item {\tt phys.at} and {\tt phys.mol} have been completely reorganized to improve the overall description 
of this domain. A new branch {\tt phys.atmol} has been introduced to group concepts shared between {\tt 
phys.at} and {\tt phys.mol}.
\item The {\tt phot.color} section was significantly simplified.
\item Missing nodes of the tree were added (e.g. {\tt em.gamma}, {\tt em.mm}, {\tt pos.sg}).
\item Creation of new words: {\tt em.wavenumber}, {\tt meta.ucd}, {\tt stat.error.sys}.
\item Typos were corrected in {\tt em.opt.*} units and a few other descriptions.
\end{enumerate}

\bibliography{ivoatex/ivoabib,ivoatex/docrepo,localrefs}
\end{document}


